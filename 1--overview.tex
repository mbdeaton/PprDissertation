\chapter{Overview}
\label{chap:overview}

\section{Why this Study?}
\label{sec:why_this_study}

Neutron stars and stellar-mass black holes form out of massive stars when they
exhaust their nuclear fuel. White dwarfs form in the same way, but from less
massive stars (with $M\lesssim 8\,M_{\odot}$), like our sun.
For stars of $M\gtrsim 8\,M_{\odot}$, generally the more massive
($M\gtrsim 20\,M_{\odot}$ end up as black holes, and the less massive
($M\lesssim 20\,M_{\odot}$) end up as neutron stars.
\citep{woos2002-review_stellar_evolution}

Unlike main-sequence stars, which are supported against gravitational collapse
by thermal pressure due to nuclear burning, neutron stars and white dwarfs
don't burn fuel; their composition is fixed.
Their resistance to collapse, in the language of quantum mechanics, is due
to degeneracy pressure, a physical manifestation of the uncertainty principle.
Neutron stars are supported by degenerate neutrons, while white dwarfs are
supported by degenerate electrons. In contrast to both of these, black holes
are the unique objects which, in equilibrium, are not supported against
collapse. They are so compact gravity overwhelms all of Nature's repulsive
forces, and their surfaces fall inside a causal horizon. After this, gravity
enforces a one-way flow of matter and radiation inward.

Both neutron stars and white dwarfs are limited by a maximum mass above which
gravitational pressure overwhelms degeneracy pressure. Neutron stars with masses
greater than $\sym2\,M_\odot$, or white dwarfs with masses greater than
$\sym1.4\,M_\odot$, cannot exist in equilibrium: they are unstable against
collapse to a black hole.

Neutron stars and black holes are called compact objects because their enclosed
mass per surface radius, or compactness $\mathcal{C}\equiv G M/c^2 R$, is very
large:
$\mathcal{C}_{\rm BH}=0.5$ and
$\mathcal{C}_{\rm NS}\sim0.15$,
whereas $\mathcal{C}_{\rm WD}\sim10^{-4}$, and $\mathcal{C}_{\odot}\sim10^{-6}$.
\todo{check $\mathcal{C}_{\rm WD}$ and $\mathcal{C}_{\odot}$}
Because gravity plays a greater role at lesser distances from a massive object,
a body's compactness is a measure of the significance of gravity
in explaining changes to that body and its immediate environment.
For neutron stars and black holes gravity is very important.

Binaries of compact objects may form by two evolutionary channels: from the
double supernovae of a pair of main sequence stars already in binary orbit,
or from dynamical capture in dense stellar regions, like globular clusters,
where near-flybys are common.
\todo{cite ... maybe Stephens+}
In either case, when two compact objects orbit each other closely, gravitational
theory predicts \citep{eins1916-integrate_field_eqns,eins1918-grav_waves}
and observations confirm \citep{will2014-review}
that their orbit slowly shrinks. Orbital energy and angular momentum
are transported away from the system by gravitational waves.

Compact binaries come in three combinations: black hole--black hole \bhbh,
neutron star--black hole \nsbh, and neutron star--neutron star \nsns.
At the writing of this thesis,
the only known compact binary systems are ten neutron star--neutron star
binaries \nsns \citep{post2014-evolution_compact_binaries}; no compact binaries
involving a black hole have yet been discovered.
Becuase all the known systems comprise neutron stars---which are radio-visible as
pulsars, blinking at exquisitely regular intervals---excellent orbital timing
information is available for them.

In fact, we can watch the gravitational drain of orbital energy in many
cases, the most famous being the Hulse-Taylor neutron star binary
\citep{huls1975-discovery}.
Since radio astronomers began tracking it in 1974, its 7.75~hr orbit has
shortened by a few milliseconds \citep{weis2010-hulse_taylor_timing}.
% Estimate based on orbital params on 52984.0 MJD (12.11.03):
% \Delta P = P_b*\dot{P_b}*24*3600 [= 70 ns per orbit]
% N        = 365.24/P_b            [= 1130 orbits per yr at end 20th Cent]
% \Delta T = N*40*\Delta P         [= -3 ms per 40 yrs at end 20th Cent]
At this rate, the two neutron stars will touch in $\sym$300~million years,
% from simple integration:
% T_merge  = P_b/\Delta T          [= 300 Myrs]
at which time the quiescent binary will go through a number of violent changes.
This is called merger.

\subsection{How Common are Neutron Star--Black Hole Mergers?}
\label{ssec:how_common}

Astrophysicists use the small collection of known neutron star--neutron star
binaries \nsns to predict the merger rate of such systems in the local universe.
From the ? known systems at the time of publication, Kalogera et al.\
\citeyearpar{kalo2004-bns_merger_rate, kalo2004-erratum} extrapolated ?
\todo{fill in}
mergers per Milky-Way-equivalent-galaxy per million years.
Neither black hole--black hole \bhbh nor neutron star--black hole \nsbh
merger rates can be extrapolated from observations.

However, at least one possible progenitor to such a system is known: Cygnus X-1.
Cygnus X-1 (discovered by ?
\todo{citealt Bowyer+ 1965}
)
is an x-ray source whose energetic and rapidly-flickering emission
is well-modeled as a $\sym15\,M_\odot$ black hole orbiting a $\sym20\,M_\odot$
main-sequence companion.
\todo{cite Wong+ 2012}
Several evolution scenarios are possible, with a small probability of forming
a compact binary in close enough orbit to merge within the age of the Universe
(about 1\% likelihood according to ?).
Dominik et al.\
\todo{citet Belczynski+ 2011}
explore these scenarios, and from their relative likelihoods, draw conclusions
about the rate of neutron star--black hole \nsbh mergers in our galaxy.
In ? million years,
\todo{check}
the companion star will exhaust its nuclear fuel and its core will collapse,
triggering a supernova. It may form a neutron star, which, if the asymmetry of
the explosion isn't so great as to gravitationally unbind it from the black hole,
will yield a neutron star--black hole \nsbh binary. In less than 1\% of the possible
scenarios, the compact binary forms close enough to merge within the age of the
Universe. If the only formation channel is through
binaries like the one that formed the Cygnus X-1 system, there must be very few
neutron star--black hole \nsbh binaries in our galaxy, and the merger rate observable
by Advanced LIGO is 1 detection every 30-250~years.
\todo{citet Belczynski+ 2011}
\todo{introduce LIGO}

%\subsubsection{Theoretical Answers}
%\label{sssc:theory}
The same astrophysicists have pursued a more comprehensive approach to this
calculation through population synthesis studies: the modeling of large numbers
of stars representative of populations observed in our own and nearby galaxies.
The models attempt to capture the particular nuclear effects, gravitational
interactions, stellar evolutions, and supernova dynamics of millions of binaries,
to reach an ultimate conclusion about the compact remnants in each system:
binary or no? and if yes, is the binary orbiting closely enough to merge in
$10^{10}$~yrs, the age of our Universe?
Population synthesis models are attended by large uncertainties,
but for Milky-Way-equivalent galaxies, at the current of the Universe,
Dominik et al. 2013\
\todo{fix cite}
predict ? neutron star--black hole \nsbh mergers per galaxy per Myr.
\todo{compared to \nsns, much more common in low-z galaxies}
This rate gives us a theoretical answer to the question, ``how common are
neutron star--black hole mergers?''.

%\subsubsection{Observational Answers}
%\label{sssc:observation}
These theoretical predictions may be complemented or challenged by observations.
In the next few years, astronomers expect to record the final seconds to minutes
\todo{check}
of the gravitational wave signal emitted by a compact binary
as it merges, using gravitational wave interferometers like LIGO, Virgo, and Geo.
\todo{cite AdvLIGO white paper}
Or if matter is at hand, that is, if one or both of the objects is a neutron
star, they may observe electromagnetic or neutrino signals. Let's discuss
electromagnetic signals here; in Sec.~\ref{ssec:neutrino_roles}, we will
turn to neutrinos.

A diverse family of electromagnetic signals is possible from a neutron
star--neutron star \nsns or neutron star--black hole merger \nsbh:
\begin{itemize}
  \item Kilonova: Some of the neutron star material may be ejected and escape
    from the system. As this neutron-rich matter expands and cools, exotic heavy
    nuclei form like ice crystals in a winter pond. These nuclei only form in a
    low-energy bath of free neutrons via rapid neutron capture, the r-process.
    Later, these unstable nuclei decay and, like a nuclear power stack,
    \todo{stack?}
    heat the ejecta, causing it to emit photons with a
    thermal spectrum, at optical or infrared frequencies.
    \citep{
      robe2011-transients,
      metz2012-most_promising,
      kase2013-opacities}.
    \todo{more cite}

    Recently space-based infrared telescopes have observed short glows that
    fit these theoretical predictions, and are promising candidates of this
    theoretically-predicted event.
    \todo{cite Berger, and the other group ~2014}
  \item Late radio afterglow: The distant neighborhood of the merger, the
    circumburst medium, is often denser than the binary's immediate environment.
    \todo{often?}
    The circumburst medium may be simply the interstellar medium.
    \todo{what else?}
    As the ejecta rams into the denser medium, it experiences hydrodynamic
    shocks, amplifying any seed magnetic fields, and causing the plasma to
    emit radio-frequency synchrotron radiation.
    \todo{synchrotron? how long? cite}
    \todo{cite observations}
  \item Early x-ray afterglow: If the progenitor is a neutron star--neutron
    star binary \nsns, the immediate product of the merger may be
    a supermassive neutron star,
    \todo{cite}
    more massive than the fundamental
    limit discussed above ($M\lesssim2\,M_\odot$), but nevertheless quasi-stable
    because it is rotating rapidly.
    If so, the supermassive neutron star may emit magnetized winds that shock
    into each other and emit thermal x-ray radiation. (Kumar-Rezolla 2015).
    \todo{fix cite, add Metzger+}
    \todo{cite observations}
  \item Gamma ray burst: If an accretion disk forms---either after the
    spin-down of the supermassive neutron star, or because a black hole forms
    immediately after merger, or because the progenitor is a neutron star--black
    hole \nsbh binary---some combination of magnetic and neutrino heating
    could drive an ultrarelativistic jet. The jet, when it shocks into the
    circumburst medium, or shocks internally, will emit high energy gamma ray
    photons, highly beamed in the direction of the jet, due to its relativistic
    speed. The burst of gamma ray emission lasts as long as the accretion is
    active enough to drive the jet: about a second.

    Many gamma ray bursts have
    been observed, and fall roughly into two categories, long and short,
    \todo{cite observations}
    according to their length. The observed short gamma ray bursts have
    properties that are well-modeled by this merger--disk--jet scenario,
    \todo{cite old proposal paper}
    and the astronomical community widely accepts the explanation.
    \todo{cite widely accepts}
    \todo{cite sgrbs}
\end{itemize}

Some of these electromagnetic signals are strong---in particular gamma ray
bursts---and our observing horizon encompasses many millions? of galaxies.
\todo{estimate this}
With space-based telescopes like Swift?, Fermi?, and INTEGRAL?
tens of short gamma ray bursts are observed per year.
\todo{check/cite Nakar 2007}
In an indirect sense, this rate gives us an observational answer to the question,
``how common are neutron star--black hole mergers?''.
But in the next few years a much more direct answer will be possible, as the
advanced generation gravitational wave interferometers, LIGO and Virgo, begin to
listen for the signature chirps of compact binary mergers. Unless we are greatly
mistaken about compact binaries, these observatories will likely pick up
? neutron star--neutron star \nsns,
? neutron star--black hole \nsbh, and
? black hole black hole \bhbh mergers per year.
\todo{cite Abadie+}

Of course it is possible the astrophysics community is mistaken.
The physical processes linking compact binary mergers to these
signals are not fully understood, an ignorance that partly motivates this thesis.
In fact, until astronomers observe one of these electromagnetic signals in
coincidence with a gravitational wave signal, it's possible (though very hard
for this physicist to imagine) that the short
gamma ray burst rate---providing the observational answer to our question
above---has nothing to do with mergers of compact binaries of any kind.

\subsubsection{Focusing in on Neutron Star--Black Hole Mergers}
Though much in the next sections is applicable to any type of merger involving
a neutron star, let us focus in on neutron star--black hole \nsbh
mergers, the topic of this thesis. Sec.~\ref{sec:why_this_model} defends this
focus more extensively than here.
But briefly: in this study, we are interested in systems involving the
strongest gravitational effects and nuclear matter. The first interest leads us
to black holes, and the second interest leads us to neutron stars.
We won't pursue neutron star--neutron star
\nsns or black hole--black hole \bhbh binaries past this section.

\subsection{What Roles May Neutrinos Play?}
\label{ssec:neutrino_roles}
Some neutron star--black hole mergers produce stellar mass accretion disks.
As matter in the disk flows from large distances to the
inner edge of the disk, where it plunges into the black hole, it loses
gravitational energy and gains kinetic and thermal energy, which escapes
as radiation. This energy transfer is immense. The velocity of gas in circular
Kepplerian orbit around mass $M_{\rm BH}$ at radius $r$ is
$v=(GM_{\rm BH}/r)^{1/2}$, and its specific energy is
\begin{eqnarray}
  e &=& \frac{1}{2}v^2-\frac{GM_{\rm BH}}{r} \nonumber \\
  &=& -\frac{GM_{\rm BH}}{2r}, \nonumber
\end{eqnarray}
where $G$ is Newton's gravitational constant.
Because the gas velocity varies with $r$, the disk shears against itself. If the
fluid is viscous (which is likely, due to turbulence, magnetic fields, or even
radiative stresses),
\todo{cite each of these}
the fast inner regions slow down, and the slow outer regions speed up. The net
effect is that gas in the inner disk moves a little closer to the black hole,
increasing its specific energy:
\begin{equation}
  \Delta e = \frac{GM_{\rm BH}}{2}
  \left(\frac{1}{r_{\rm final}}-\frac{1}{r_{\rm initial}}\right) \nonumber
\end{equation}
This process transports gas inward until it plunges into the black hole.
If most of the disk accretes, the total increase in the energy of the gas
is $E_{\rm acc}\sim M_{\rm disk} \Delta e$. How much is that?

We may estimate $r_{\rm initial}$ by the radius of the disk. Simulations give
\todo{cite}
characteristic scales of:
\begin{eqnarray}
  R_{\rm disk} &\sim& 100\,{\rm km} \nonumber \\
  M_{\rm disk} &\sim& 0.01\text{--}0.5\,M_\odot. \nonumber
\end{eqnarray}
And we may estimate the inner edge of the disk by
\begin{eqnarray}
  r_{\rm final} &\sim& 1\text{--}9 \times \frac{GM_{\rm BH}}{c^2} \nonumber \\
  &\sim& 1.5\text{--}13.5 \,{\rm km}\times \left(\frac{M_{\rm BH}}{M_\odot}\right), \nonumber
\end{eqnarray}
the radius of last stable circular orbit around the black hole. The range of this
radius represents its dependence on the black hole's spin: the higher the spin
in the direction of the disk's rotation, the closer may the gas approach before it
plunges into the black hole.
Thus the gravitational energy released by a stellar-mass accretion disk is
\begin{equation}
  E_{\rm acc} \sim 10^{51\text{--}54} \,{\rm erg}.
\end{equation}
This energy is initially transferred into the bulk motion of the gas, as it
spirals in toward the black hole. But turbulence and viscosity transfer it to
smaller scales: ordered kinetic energy becomes disordered thermal energy.
The disk heats up and radiates this energy at a supernova-scale luminosity:
\begin{equation}
  L \sim \frac{E_{\rm acc}}{100\,{\rm ms}}
  \sim 10^{52\text{--}55}
  \,{\rm erg \, s}^{-1}, \nonumber
\end{equation}
where we have used an accretion timescale of 100~ms, as determined by simulations.
\todo{cite}
Surprisingly, unlike most hot objects in the universe, the enormous energy output
of an accretion disk is not carried away by photons.

Photons scatter off of free ions and over a characteristic length of
$\lambda_\gamma\sim(Y_e n_b\sigma_{\rm T})^{-1}$, where $Y_e$ is the local number
ratio of electrons to baryons, $n_b$ is the baryon number density, and
$\sigma_{\rm T}\approx6.6\times10^{-25}\,{\rm cm}^2$ is the cross-section
for photons scattering elastically off of electrons. The disk being slightly
neutron-rich, $Y_e$ is in the range of 0.1. And typical densities are a
few orders of magnitude less than the central density of a neutron star,
or $\rho\sim10^{11}\,{\rm g\,cm}^{-3}$,
so that $n_b=\rho/m_U\approx10^{-4}\,{\rm fm}^{-3}$.
A disk's photon optical depth (that is, the square root of the number of times
a photons scatters before it leaves the disk) is of order
\begin{eqnarray}
  \tau_\gamma &\sim& \frac{R_{\rm disk}}{\lambda_\gamma} \nonumber \\
  &\approx& (10^7\,{\rm cm})(0.1)(10^{35}\,{\rm cm}^{-3})(10^{-24}\,{\rm cm}^2) \nonumber \\
  &\approx& 10^{17}
\end{eqnarray}
Photons can't carry away the thermal energy of the disk: they're trapped in it.

%********************************************************************************
% Remnant stuff from pre-first-draft

However, neutrinos are freely produced in the neutron-rich matter. The
interactions that dominate neutrino opacity are scattering and absorption onto
nucleons and nuclei \citep[Sec.\ 11.7]{shap1983-bh_wd_ns}, all of which processes
scale like $\sigma_0(\varepsilon/m_e c^2)^2$. Here $\sigma_0$ is the weak
scattering cross section,
\begin{align}
  \sigma_0
  &\equiv \frac{4}{\pi}\left(\frac{\hbar}{m_e c}\right)^{-4}
  \left(\frac{G_F}{m_e c^2}\right)^2 \\
  &\sim   1.76 \times 10^{-44} \,\, {\rm cm}^2.
\end{align}
(See, for example, \citealt{tubb1975-neutrino_opacities}.)
%********************************************************************************

\section{What's the History Behind this Study?}
\label{sec:history}

\section{Why this Particular Model?}
\label{sec:why_this_model}

\section{What's the Scope and Trajectory of this Thesis?}
\label{sec:scope}
