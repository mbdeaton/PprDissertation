\chapter{Overview}
\label{chap:overview}

\section{Why this Study?}

Neutron stars and stellar-mass black holes form out of massive stars when they
exhaust their nuclear fuel. White dwarfs form in the same way, but from less
massive stars (with $M\lesssim 8$~M$_{\odot}$), like our sun.
For stars of $M\gtrsim 8$~M$_{\odot}$, generally the more massive
($M\gtrsim 20$~M$_{\odot}$ end up as black holes, and the less massive
($M\lesssim 20$~M$_{\odot}$) end up as neutron stars.
\citep{woos2002-review_stellar_evolution}

Unlike main-sequence stars, which are supported against gravitational collapse
by thermal pressure due to nuclear burning, neutron stars and white dwarfs
don't burn fuel; their composition is fixed.
Their resistance to collapse, in the language of quantum mechanics, is due
to degeneracy pressure, a physical manifestation of the uncertainty principle.
Neutron stars are supported by degenerate neutrons, while white dwarfs are
supported by degenerate electrons. In contrast to both of these, black holes
are the unique objects which, in equilibrium, are not supported against
collapse. They are so compact gravity overwhelms all of Nature's repulsive
forces, and their surfaces fall inside a causal horizon. After this gravity
enforces a one-way flow of matter and radiation inward.

Both neutron stars and white dwarfs are limited by a maximum mass above which
gravitational pressure overwhelms degeneracy pressure. Neutron stars with masses
greater than $\sym$2~M$_\odot$, or white dwarfs with masses greater than
$\sym$1.4~M$_\odot$, cannot exist in equilibrium: they are unstable against
collapse to a black hole.

All of these remnants are called compact objects because their enclosed mass per
surface radius, or compactness $\mathcal{C}\equiv G M/c^2 R$, is very large:
$\mathcal{C}_{\rm BH}=0.5$,
$\mathcal{C}_{\rm NS}\sim0.15$,
$\mathcal{C}_{\rm WD}\sim10^{-4}$,
whereas $\mathcal{C}_{\odot}\sim10^{-6}$.
\todo{check $\mathcal{C}_{\rm WD}$ and $\mathcal{C}_{\odot}$}
Because gravity plays a greater role at smaller distances from a massive object,
a body's compactness is a measure of the significance of gravitational theory
in explaining changes to that body and its immediate environment.
For neutron stars and black holes gravity is very important.

In some cases binaries of compact objects form from binaries of massive stars,
pairs of stars in orbit around one another since formation. Also in regions of
space densely packed with stars, like globular clusters, where near-flybys are
common, compact object binaries can form by dynamical capture.
\todo{cite ... maybe Stephens et al.}
In either case, when two compact objects orbit each other closely gravitational
theory predicts their orbit slowly shrinks. Orbital energy and angular momentum
are transported away from the system by gravitational waves.

Compact binaries come in three combinations: black hole--black hole, neutron
star--black hole, and neutron star--neutron star. At the writing of this thesis,
? neutron star--neutron star binaries have been discovered in our galaxy,
\todo{check/cite}
but no compact binaries involving a black hole.
Becuase the known systems have neutron stars, which are radio-visible as
pulsars, blinking at exquisitely regular intervals, excellent orbital timing
information is available for them.

In fact, we can watch the gravitational drain of orbital energy in in many
cases, the most famous being the Hulse-Taylor neutron star binary
\citep{huls1974-discovery}.
Since radio astronomers began tracking it in 197?, its ? hour orbit has
shortened by ? ms.
At this rate, the two neutron stars will touch in $\sym$10~billion years,
\todo{check/cite}
at which time the quiescent binary will go through a number of violent changes.
This is called merger.

\subsection{How common are neutron star--black hole mergers?}

\subsection{What roles may neutrinos play?}

%********************************************************************************
% Remnant stuff from pre-first-draft

However, neutrinos are freely produced in the neutron-rich matter. The
interactions that dominate neutrino opacity are scattering and absorption onto
nucleons and nuclei \citep[Sec.\ 11.7]{shap1983-bh_wd_ns}, all of which processes
scale like $\sigma_0(\varepsilon/m_e c^2)^2$. Here $\sigma_0$ is the weak
scattering cross section,
\begin{align}
  \sigma_0
  &\equiv \frac{4}{\pi}\left(\frac{\hbar}{m_e c}\right)^{-4}
  \left(\frac{G_F}{m_e c^2}\right)^2 \\
  &\sim   1.76 \times 10^{-44} \,\, {\rm cm}^2.
\end{align}
(See, for example, \citealt{tubb1975-neutrino_opacities}.)
%********************************************************************************

\section{What's the History Behind this Study?}

\section{Why this Particular Model?}

\section{What's the Scope and Trajectory of this Thesis?}
