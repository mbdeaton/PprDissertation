\chapter{Overview}
\label{chap:overview}

\section{Why this Study?}

Neutron stars and stellar-mass black holes form out of massive stars when they
exhaust their nuclear fuel. White dwarfs form in the same way, but from less
massive stars (with $M\lesssim 8$~M$_{\odot}$), like our sun.
For stars of $M\gtrsim 8$~M$_{\odot}$, generally the more massive
($M\gtrsim 20$~M$_{\odot}$ end up as black holes, and the less massive
($M\lesssim 20$~M$_{\odot}$) end up as neutron stars.
\citep{woos2002-review_stellar_evolution}

Unlike main-sequence stars, which are supported against gravitational collapse
by thermal pressure due to nuclear burning, neutron stars and white dwarfs
don't burn fuel; their composition is fixed.
Their resistance to collapse, in the language of quantum mechanics, is due
to degeneracy pressure, a physical manifestation of the uncertainty principle.
Neutron stars are supported by degenerate neutrons, while white dwarfs are
supported by degenerate electrons. In contrast to both of these, black holes
are the unique objects which, in equilibrium, are not supported against
collapse. They are so compact gravity overwhelms all of Nature's repulsive
forces, and their surfaces fall inside a causal horizon. After this gravity
enforces a one-way flow of matter and radiation inward.

Both neutron stars and white dwarfs are limited by a maximum mass above which
gravitational pressure overwhelms degeneracy pressure. Neutron stars with masses
greater than $\sym$2~M$_\odot$, or white dwarfs with masses greater than
$\sym$1.4~M$_\odot$, cannot exist in equilibrium: they are unstable against
collapse to a black hole.

All of these remnants are called compact objects because their enclosed mass per
surface radius, or compactness $\mathcal{C}\equiv G M/c^2 R$, is very large:
$\mathcal{C}_{\rm BH}=0.5$,
$\mathcal{C}_{\rm NS}\sim0.15$,
$\mathcal{C}_{\rm WD}\sim10^{-4}$,
whereas $\mathcal{C}_{\odot}\sim10^{-6}$.
\todo{check $\mathcal{C}_{\rm WD}$ and $\mathcal{C}_{\odot}$}
Because gravity plays a greater role at smaller distances from a massive object,
a body's compactness is a measure of the significance of gravity
in explaining changes to that body and its immediate environment.
For neutron stars and black holes gravity is very important.

Binaries of compact objects may form by two evolutionary channels: from the
double supernovae of a pair of main sequence stars already in binary orbit,
or from dynamical capture in dense stellar regions, like globular clusters,
where near-flybys are common.
\todo{cite ... maybe Stephens+}
In either case, when two compact objects orbit each other closely, gravitational
theory predicts their orbit slowly shrinks. Orbital energy and angular momentum
are transported away from the system by gravitational waves.

Compact binaries come in three combinations: black hole--black hole, neutron
star--black hole, and neutron star--neutron star. At the writing of this thesis,
the only known compact binary systems are 10 neutron star--neutron star binaries
\citep{post2014-evolution_compact_binaries}; no compact binaries involving a
black hole have yet been discovered.
Becuase all the known systems comprise neutron stars, which are radio-visible as
pulsars, blinking at exquisitely regular intervals, excellent orbital timing
information is available for them.

In fact, we can watch the gravitational drain of orbital energy in many
cases, the most famous being the Hulse-Taylor neutron star binary
\citep{huls1975-discovery}.
Since radio astronomers began tracking it in 1974, its 7.75~hr orbit has
shortened by a few milliseconds \citep{weis2010-hulse_taylor_timing}.
% Estimate based on orbital params on 52984.0 MJD (12.11.03):
% \Delta P = P_b*\dot{P_b}*24*3600 [= 70 ns per orbit]
% N        = 365.24/P_b            [= 1130 orbits per yr at end 20th Cent]
% \Delta T = N*40*\Delta P         [= -3 ms per 40 yrs at end 20th Cent]
At this rate, the two neutron stars will touch in $\sym$300~million years,
% from simple integration:
% T_merge  = P_b/\Delta T          [= 300 Myrs]
at which time the quiescent binary will go through a number of violent changes.
This is called merger.

\subsection{How common are neutron star--black hole mergers?}

Astrophysicists use the small collection of known neutron star--neutron star
binaries to predict the merger rate of such systems in the local universe.
From the ? known systems at the time of publication, Kalogera et al.\
\citeyearpar{kalo2004-bns_merger_rate, kalo2004-erratum} extrapolate ?
\todo{fill in}
mergers per Milky-Way-equivalent-galaxy per million years.
Neither black hole--black hole or neutron star--black hole merger rates cannot
be extrapolated from observations.

However, one possible progenitor to such a system is known: Cygnus X-1.
Cygnus X-1 (discovered by ?
\todo{citealt Bowyer+ 1965}
)
is an x-ray source whose energetic and rapidly-flickering emission
is well-modeled as a 15~M$_\odot$ black hole orbiting a 19~M$_\odot$
main-sequence companion.
\todo{cite Wong+ 2012}
Several evolution scenarios are possible, with a small probability of forming
a compact binary in close enough orbit to merge within the age of the Universe
(about 1\% according to ?).
Dominik et al.\
\todo{citet Belczynski+ 2011}
explore these scenarios, and from their relative likelihoods, draw conclusions
about the rate of neutron star--black hole mergers in our galaxy.
In ? million years,
\todo{check}
the companion star will exhaust its nuclear fuel and its core will collapse,
triggering a supernova. It may form a neutron star, which, if the asymmetry of
the explosion isn't so great as to gravitationally unbind it from the black hole,
will yield a neutron star--black hole binary. In less than 1\% of the possible
scenarios, the compact binary forms close enough to merge within the age of the
Universe. If the only formation channel is through
binaries like the one that formed the Cygnus X-1 system, there must not be many
neutron star--black hole binaries in our galaxy, and the merger rate observable
by Advanced LIGO is 1 detection every 30-250~years.
\todo{citet Belczynski+ 2011}

The same astrophysicists have pursued a more comprehensive approach to this
calculation through population synthesis studies: the modeling of large numbers
of stars representative of populations observed in our own and nearby galaxies.
The models are driven by nuclear physics, gravitational interactions, and
supernova models to reach an ultimate compact object conclusion: binary or no,
and if yes, then is it orbiting closely enough to merge withing the age of
our Universe? The models are attended by large uncertainties, but they
predict ? neutron star--black hole mergers per Milky-Way-equivalent-galaxy
per Myr.
\todo{check/cite Dominik+ 2013}

\subsection{What roles may neutrinos play?}

%********************************************************************************
% Remnant stuff from pre-first-draft

However, neutrinos are freely produced in the neutron-rich matter. The
interactions that dominate neutrino opacity are scattering and absorption onto
nucleons and nuclei \citep[Sec.\ 11.7]{shap1983-bh_wd_ns}, all of which processes
scale like $\sigma_0(\varepsilon/m_e c^2)^2$. Here $\sigma_0$ is the weak
scattering cross section,
\begin{align}
  \sigma_0
  &\equiv \frac{4}{\pi}\left(\frac{\hbar}{m_e c}\right)^{-4}
  \left(\frac{G_F}{m_e c^2}\right)^2 \\
  &\sim   1.76 \times 10^{-44} \,\, {\rm cm}^2.
\end{align}
(See, for example, \citealt{tubb1975-neutrino_opacities}.)
%********************************************************************************

\section{What's the History Behind this Study?}

\section{Why this Particular Model?}

\section{What's the Scope and Trajectory of this Thesis?}
