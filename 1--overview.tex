\chapter{Overview}
\label{chap:overview}

Neutron stars and stellar-mass black holes form out of massive stars when they
exhaust their nuclear fuel. White dwarfs form in the same way, but from less
massive stars, with $M\lesssim 8\,M_{\odot}$, like our sun.
For stars of $M\gtrsim 8\,M_{\odot}$, generally the more massive
($M\gtrsim 20\,M_{\odot}$ end up as black holes, and the less massive
($M\lesssim 20\,M_{\odot}$) end up as neutron stars
\citep{woos2002-review_stellar_evolution}.

Unlike main-sequence stars, which are supported against gravitational collapse
by thermal pressure due to nuclear burning, neutron stars and white dwarfs
don't burn fuel; their composition is fixed.
Their resistance to collapse, in the language of quantum mechanics, is due
to degeneracy pressure, a physical manifestation of the uncertainty principle.
Neutron stars are supported by degenerate neutrons, while white dwarfs are
supported by degenerate electrons. In contrast to both of these, black holes
are the unique objects which, in equilibrium, are not supported against
collapse. They are so compact gravity overwhelms all of Nature's repulsive
forces, and their surfaces fall inside a causal horizon. After this, gravity
enforces a one-way flow of matter and radiation inward.

Both neutron stars and white dwarfs are limited by a maximum mass above which
gravitational pressure overwhelms degeneracy pressure. Neutron stars with masses
greater than $\sym2\,M_\odot$, or white dwarfs with masses greater than
$\sym1.4\,M_\odot$, cannot exist in equilibrium: they are unstable against
collapse to a black hole.

Neutron stars and black holes are called compact objects because their enclosed
mass per surface radius, or compactness $\mathcal{C}\equiv G M/c^2 R$, is very
large:
$\mathcal{C}_{\rm BH}=0.5$ and
$\mathcal{C}_{\rm NS}\sim0.15$,
whereas $\mathcal{C}_{\rm WD}\sim10^{-4}$, and $\mathcal{C}_{\odot}\sim10^{-6}$.
\todo{check $\mathcal{C}_{\rm WD}$ and $\mathcal{C}_{\odot}$}
Because gravity plays a greater role at lesser distances from a massive object,
a body's compactness is a measure of the significance of gravity
in explaining changes to that body and its immediate environment.
For neutron stars and black holes, gravity is very important.

Binaries of compact objects may form by two evolutionary channels: from the
double supernovae of a pair of main sequence stars already in binary orbit,
or from dynamical capture in dense stellar regions, like globular clusters,
where near-flybys are common.
\todo{cite ... maybe Stephens+}
In either case, when two compact objects orbit each other closely, gravitational
theory predicts \citep{eins1916-integrate_field_eqns,eins1918-grav_waves}
and observations confirm \citep{will2014-review}
that their orbit slowly shrinks. Orbital energy and angular momentum
are transported away from the system by gravitational waves.

Compact binaries come in three combinations: black hole--black hole \bhbh,
neutron star--black hole \nsbh, and neutron star--neutron star \nsns.
At the writing of this thesis,
the only known compact binary systems are ten neutron star--neutron star
\nsns binaries \citep{post2014-evolution_compact_binaries}; no compact binaries
involving a black hole have yet been discovered.
Becuase all the known systems comprise neutron stars---which are radio-visible as
pulsars, blinking at exquisitely regular intervals---excellent orbital timing
information is available for them.

In fact, we can watch the gravitational drain of orbital energy in many
cases, the most famous being the Hulse-Taylor neutron star--neutron star \nsns
binary \citep{huls1975-discovery}.
Since radio astronomers began tracking it in 1974, its 7.75~hr orbit has
shortened by a few milliseconds \citep{weis2010-hulse_taylor_timing}.
% Estimate based on orbital params on 52984.0 MJD (12.11.03):
% \Delta P = P_b*\dot{P_b}*24*3600 [= 70 ns per orbit]
% N        = 365.24/P_b            [= 1130 orbits per yr at end 20th Cent]
% \Delta T = N*40*\Delta P         [= -3 ms per 40 yrs at end 20th Cent]
At this rate, the two neutron stars will touch in $\sym$300~million years,
% from simple integration:
% T_merge  = P_b/\Delta T          [= 300 Myrs]
at which time the quiescent binary will go through a number of violent changes.
This is called merger.

\section{Are Neutron Star--Black Hole Mergers Common?}
\label{sec:how_common}

Astrophysicists use the small collection of known neutron star--neutron star
binaries \nsns to predict the merger rate of such systems in the local universe.
From the ? known systems at the time of publication, Kalogera et al.\
\citeyearpar{kalo2004-bns_merger_rate, kalo2004-erratum} extrapolated ?
\todo{fill in}
mergers per Milky-Way-equivalent-galaxy per million years.
Neither black hole--black hole \bhbh nor neutron star--black hole \nsbh
merger rates can be extrapolated from observations.

However, at least one possible progenitor to a neutron star--black hole \nsbh
binary is known: Cygnus X-1. Cygnus X-1 (discovered by ?)
\todo{citealt Bowyer+ 1965}
is an x-ray source whose energetic and rapidly-flickering emission
is well-modeled as a $\sym15\,M_\odot$ black hole orbiting a $\sym20\,M_\odot$
main-sequence companion.
\todo{cite Wong+ 2012}
Several evolution scenarios are possible, with a small probability of forming
a compact binary in close enough orbit to merge within the age of the Universe
(about 1\% likelihood according to ?).
\citet{belc2011-cyg_x1}
explore these scenarios, and from their relative likelihoods, draw conclusions
about the rate of neutron star--black hole \nsbh mergers in our galaxy.
In ? million years,
\todo{check}
the companion star will exhaust its nuclear fuel and its core will collapse,
triggering a supernova. It may form a neutron star, which, if the asymmetry of
the explosion isn't so great as to gravitationally unbind it from the black hole,
will yield a neutron star--black hole \nsbh binary. In less than 1\% of the possible
scenarios, the compact binary forms close enough to merge within the age of the
Universe. If the only formation channel is through
binaries like the one that formed the Cygnus X-1 system, there must be very few
neutron star--black hole \nsbh binaries in our galaxy, and the merger rate observable
by Advanced LIGO is 1 detection every 30-250~years.
\todo{check}
\todo{introduce LIGO}

%\subsubsection{Theoretical Answers}
%\label{sssc:theory}
The same astrophysicists have pursued a more comprehensive approach to this
calculation through population synthesis studies: the modeling of large numbers
of stars representative of populations observed in our own and nearby galaxies.
The models attempt to capture the particular nuclear effects, gravitational
interactions, stellar evolutions, and supernova dynamics of millions of binaries,
to reach an ultimate conclusion about the compact remnants in each system:
binary or no? and if yes, is the binary orbiting closely enough to merge in
$10^{10}$~yrs, the age of our Universe?
Population synthesis models are attended by large uncertainties
(see especially \citealt{domi2012-merger_rates_part1}).
But one attempt to bracket these uncertainties
has revealed that at the current age of the Universe,
field populations of neutron star--black hole \nsbh binaries are probably
merging at a rate between $\ll1$ and $\sym10$
per year per Gpc$^3$, with the most likely rates near 1~yr$^{-1}$~Gpc$^{-3}$
\citep[Figs.~3 and~5]{domi2013-merger_rates_part2}.
According to this study, and with good agreement between a range of models,
the local neutron star--black hole merger rate peaked long ago,
around $z\sim3$, as high as $50\text{--}100$~yr$^{-1}$~Gpc$^{-3}$.
\todo{compared to \nsns, much more common in low-z galaxies}
This range of rates gives us a theoretical answer to the question,
``how common are neutron star--black hole mergers?''.

%\subsubsection{Observational Answers}
%\label{sssc:observation}
These theoretical predictions may be complemented or challenged by observations.
In the next few years, astronomers expect to record the final seconds to minutes
\todo{check}
of the gravitational wave signal emitted by a compact binary
as it merges, using gravitational wave interferometers like LIGO, Virgo, and Geo.
\todo{cite AdvLIGO white paper}
Or if matter is at hand, that is, if one or both of the objects is a neutron
star, they may observe electromagnetic or neutrino signals. Let's discuss
electromagnetic signals here; in Sec.~\ref{ssec:neutrino_roles}, we will
turn to neutrinos.

A diverse family of electromagnetic signals is possible from a neutron
star--neutron star \nsns or neutron star--black hole merger \nsbh:
\begin{itemize}
  \item Kilonova: Some of the neutron star material may be ejected and escape
    from the system. As this neutron-rich matter expands and cools, exotic heavy
    nuclei form like ice crystals in a winter pond. These nuclei only form in a
    low-energy bath of free neutrons via rapid neutron capture, the r-process.
    Later, these unstable nuclei decay and, like a nuclear power reactor,
    heat the ejecta, causing it to emit photons with a
    thermal spectrum, at optical or infrared frequencies.
    \citep{
      li1998-transients,
      robe2011-transients,
      metz2012-most_promising,
      kase2013-opacities}.

    Recently space-based infrared telescopes have observed short glows that
    fit these theoretical predictions, and are promising candidates of this
    theoretically-predicted event.
    \todo{cite Berger, and the other group ~2014}
  \item Late radio afterglow: The distant neighborhood of the merger, the
    circumburst medium, is often denser than the binary's immediate environment.
    \todo{often?}
    The circumburst medium may be simply the interstellar medium.
    \todo{what else?}
    As the ejecta rams into the denser medium, it experiences hydrodynamic
    shocks, amplifying any seed magnetic fields, and causing the plasma to
    emit radio-frequency synchrotron radiation.
    \todo{synchrotron? how long? cite}
    \todo{cite observations}
  \item X-rays emission: If the progenitor is a neutron star--neutron
    star binary \nsns, the immediate product of the merger may be
    a supermassive neutron star,
    \todo{cite}
    more massive than the fundamental
    limit discussed above ($M\lesssim2\,M_\odot$), but nevertheless quasi-stable
    because it is rotating rapidly.
    If so, the supermassive neutron star may emit winds that shock
    into each other and emit thermal x-ray radiation \citep{rezz2015-two_winds}.
    Afterglows have been detected \citep{fong2012-xray_afterglow}, and in fact,
    so have x-ray precursors \citep{troj2010-sgrb_precursors}, which are
    not well-understood. One possible source for an x-ray precursor from either
    neutron star--neutron star \nsns or neutron star--black hole \nsbh mergers is
    the shattering of the neutron star's crust due to tidal forces
    \citep{tsan2012-shattering}.
  \item Gamma ray burst: If an accretion disk forms---either after the
    spin-down of the supermassive neutron star, or because a black hole forms
    immediately after merger, or because the progenitor is a neutron star--black
    hole \nsbh binary---some combination of magnetic and neutrino heating
    could drive an ultrarelativistic jet. The jet, when it shocks into the
    circumburst medium, or shocks internally, will emit high energy gamma ray
    photons, highly beamed in the direction of the jet, due to its relativistic
    speed. The burst of gamma ray emission lasts as long as the accretion is
    active enough to drive the jet: about a second.

    Many gamma ray bursts have
    been observed, and fall roughly into two categories, long and short,
    \todo{cite observations}
    according to their length. The observed short gamma ray bursts have
    properties that are well-modeled by this merger--disk--jet scenario,
    \todo{cite old proposal paper}
    and the astronomical community widely accepts the explanation.
    \todo{cite widely accepts}
    \todo{cite sgrbs}
\end{itemize}
Some of these electromagnetic signals are strong---in particular gamma ray
bursts---and our observing horizon encompasses many millions? of galaxies.
\todo{estimate this}
With space-based telescopes like Swift?, Fermi?, and INTEGRAL?
tens of short gamma ray bursts are observed per year.
\todo{check/cite Nakar 2007}
In an indirect sense, this rate gives us an observational answer to the question,
``how common are neutron star--black hole mergers?''.

But in the next few years a much more direct answer will be possible, as the
advanced generation gravitational wave interferometers, Advanced LIGO and Virgo,
begin to record and search gravitational strain data for the signature chirps of
compact binary mergers. Unless astrophysicsists are fundamentally
mistaken about the formation of compact binaries, the three-detector network of
Advanced LIGO's two instruments plus Advanced Virgo, when operating at design
sensitivity, will likely detect neutron star--black hole \nsbh mergers at a
rate of 0.1--10~yr$^{-1}$
\citep[Tables~2 and~3]{domi2014-merger_rates_part3}.\footnote{
I report here the most significant figure of the brackets given by their range
of models, assuming only inspiral waveforms are used in matched filter searches,
and that the network signal-to-noise ratio is set to a threshold of 10.
Their predictions for neutron star--neutron star \nsns
and black hole--black hole \bhbh merger detections are
2--7~yr$^{-1}$ and 4--2000~yr$^{-1}$.}
\footnote{
These results are unpublished at the writing of this thesis.
However, traditionally, estimates of merger rates have been reported from the
\citet{abad2010-rates} review, which drew heavily on less-sophisticated versions
of the population synthesis models used by \citet{domi2014-merger_rates_part3}.}

Of course it is possible the astrophysics community is mistaken.
The physical processes linking compact binary mergers to these
signals are not fully understood, an ignorance that partly motivates this thesis.
In fact, until astronomers observe one of these electromagnetic signals in
coincidence with a gravitational wave signal, it's possible (though very hard
for this physicist to imagine) that the short
gamma ray burst rate---providing the observational answer to our question
above---has nothing to do with mergers of compact binaries of any kind.

Though much in the next sections is applicable to any type of merger involving
a neutron star, let us focus in on neutron star--black hole \nsbh
mergers, the topic of this thesis. Sec.~\ref{sec:why_this_model} defends this
focus more extensively than here.
But briefly: in this study, we are interested in systems involving the
strongest gravitational effects and nuclear matter. The first interest leads us
to black holes, and the second interest leads us to neutron stars.
We won't pursue neutron star--neutron star
\nsns or black hole--black hole \bhbh binaries past this section.

\section{What Roles May Neutrinos Play?}
\label{sec:neutrino_roles}
Some neutron star--black hole mergers produce stellar mass accretion disks.
As matter in the disk flows from large distances to the
inner edge of the disk, where it plunges into the black hole, it loses
gravitational energy and gains kinetic and thermal energy, which escapes
as radiation. This energy transfer is immense. The velocity of gas in circular
Kepplerian orbit around mass $M_{\rm BH}$ at radius $r$
(easy to remember by the 1-2-3 rule: $GM_{\rm BH}^1=\Omega^2r^3$) is
\begin{equation}
  v = \Omega r  = (GM_{\rm BH}/r)^{1/2}, \nonumber
\end{equation}
where $G$ is Newton's gravitational constant.
Its specific energy is therefore
\begin{align}
  e &= \frac{1}{2}v^2-\frac{GM_{\rm BH}}{r} \nonumber \\
  &= -\frac{GM_{\rm BH}}{2r}. \nonumber
\end{align}
Because the gas velocity varies with $r$, the disk shears against itself. If the
fluid is viscous (which is likely, due to turbulence, magnetic fields, or even
radiative stresses),
\todo{cite each of these}
the fast inner regions slow down, and the slow outer regions speed up. The net
effect is that gas in the inner disk moves a little closer to the black hole,
increasing its specific energy:
\begin{equation}
  \Delta e = \frac{GM_{\rm BH}}{2}
  \left(\frac{1}{r_{\rm final}}-\frac{1}{r_{\rm initial}}\right). \nonumber
\end{equation}
This process transports gas inward until it plunges into the black hole.
If most of the disk accretes, the total increase in the energy of the gas
is $E_{\rm acc}\sim M_{\rm disk} \Delta e$. How much is that?

We may estimate $r_{\rm initial}$ by the radius of the disk. Simulations give
\todo{cite}
characteristic scales of:
\begin{align}
  R_{\rm disk} &\sim 100\,{\rm km} \nonumber \\
  M_{\rm disk} &\sim 0.01\text{--}0.5\,M_\odot. \nonumber
\end{align}
And we may estimate the inner edge of the disk by
\begin{eqnarray}
  r_{\rm final} &\sim& 1\text{--}9 \times \frac{GM_{\rm BH}}{c^2} \nonumber \\
  &\sim& 1.5\text{--}13.5 \,{\rm km}\times \left(\frac{M_{\rm BH}}{M_\odot}\right), \nonumber
\end{eqnarray}
the radius of last stable circular orbit around the black hole. The range of this
radius represents its dependence on the black hole's spin: the higher the spin
in the direction of the disk's rotation, the closer may the gas approach before it
plunges into the black hole.
Thus the gravitational energy released by a stellar-mass accretion disk is
\begin{equation}
  E_{\rm acc} \sim 10^{51\text{--}54} \,{\rm erg}. \nonumber
\end{equation}
This energy is initially transferred into the bulk motion of the gas, as it
spirals in toward the black hole. But turbulence and viscosity transfer it to
smaller scales: ordered kinetic energy becomes disordered thermal energy.
The disk heats up and radiates this energy at a supernova-scale luminosity:
\begin{equation}
  \label{eqn:L_order_mag}
  L \sim \frac{E_{\rm acc}}{100\,{\rm ms}}
  \sim 10^{52\text{--}55}
  \,{\rm erg \, s}^{-1},
\end{equation}
where we have used an accretion timescale of 100~ms, as determined by simulations.
\todo{cite}
Surprisingly, unlike most hot objects in the universe, the enormous energy output
of an accretion disk is not carried away by photons.
\todo{mention significance of compactness in this analysis}

\subsubsection{Radiative Energy Transport}
\label{sssc:nu_E_transport}
Photons scatter off of free ions and over a characteristic length of
$\lambda_\gamma\sim(2Y_e n_b\sigma_{\rm T})^{-1}$, where $2Y_e$ is the local
number ratio of ions to baryons, $n_b$ is the baryon number density, and
$\sigma_{\rm T}=6.6\times10^{-25}\,{\rm cm}^2$ is the cross section
for photons scattering elastically off of electrons. The disk being
neutron-rich, $Y_e$ is in the range of 0.1. And typical densities are a
few orders of magnitude less than the central density of a neutron star,
or $\rho\sim10^{11}\,{\rm g\,cm}^{-3}$,
so that $n_b=\rho/m_U\approx10^{-4}\,{\rm fm}^{-3}$.
A disk's photon optical depth (that is, the square root of the number of times
a photon scatters before it leaves the disk) is of order
\begin{align}
  \tau_\gamma &\sim \frac{R_{\rm disk}}{\lambda_\gamma} \nonumber \\
  &\approx (10^7\,{\rm cm})(0.1)(10^{35}\,{\rm cm}^{-3})(10^{-24}\,{\rm cm}^2)
  \approx 10^{17}, \nonumber
\end{align}
(where we have disregarded factors close to 1)
and the timescale for a photon to escape is
$T_{\gamma,\rm esc}\sim\tau_\gamma^2\lambda_\gamma/c\approx1$~Myr, or ten
trillion times longer than the lifetime of the disk.
Photons can't carry away the thermal energy of the disk: they're trapped in it.

But neutrinos can. Neutrinos are freely produced in the neutron-rich matter.
The dominant weak interactions contributing to their scattering
are proportional (times a numerical factor close to 1)
\todo{too simplistic}
to the weak scattering cross section:
\begin{equation}
  \sigma_0
  \equiv \frac{4}{\pi}\left(\frac{\hbar}{m_e c}\right)^{-4}
  \left(\frac{G_F}{m_e c^2}\right)^2
  \approx 1.8 \times 10^{-44} \,\, {\rm cm}^2, \nonumber
\end{equation}
and scale with neutrino energy like $(\varepsilon_\nu/m_e c^2)^2$
\citep{tubb1975-neutrino_opacities,shap1983-bh_wd_ns}.
Neutrinos scatter primarily off of the free nucleons in the disk, so their
characteristic scattering length is $\lambda_\nu\sim(n_b\sigma_0)^{-1}$,
where $Y_e$ is the local number ratio of electrons to baryons.
A disk's typical neutrino optical depth is therefore
\begin{align}
  \tau_\nu
  &\sim \tau_\gamma \frac{\sigma_0}{\sigma_T} \left(\frac{\varepsilon_\nu}{m_e c^2}\right)^2 \nonumber \\
  &\approx 0.01\times\left(\frac{\varepsilon_\nu}{\rm MeV}\right)^2. \nonumber
\end{align}
In the post-merger accretion disk, fluid temperatures are $\sym10$~MeV (which
gives us the name `nuclear' accretion disk, since temperatures are comparable to
nuclear binding energies), and $\tau_\nu\sim1$.
Neutrinos are neither totally trapped nor totally free: they play a
significant role in energy transport in the disk.
In fact, simulations of nuclear accretion disks, with realistic neutrino
treatments (see Sec.~\ref{sec:history} and Chap.~\ref{chap:leakage}) yield neutrino
luminosities comparable to the power estimate from Eqn.~\ref{eqn:L_order_mag},
indicating that neutrinos serve to drain the gravitational energy released by
accretion.

\subsubsection{Composition Changes}
\label{sssc:composition}
But neutrinos don't just transport energy in and away from the accretion disk.
Because they carry lepton charge, neutrino absorptions change fluid composition
via the charged-current reactions:
\begin{align}
  \label{eqn:beta_n_to_p}
  \nu_e \, n \rightarrow e^{-} \, p       & \qquad {\rm increasing}\,Y_e, \\
  \label{eqn:beta_p_to_n}
  \bar{\nu}_e \, p \rightarrow e^{+} \, n & \qquad {\rm decreasing}\,Y_e.
\end{align}
where $\nu_e$ and $\bar{\nu}_e$ are electron-type neutrino and antineutrino,
$e^{-}$ and $e^{+}$ are electron and positron, and $n$ and $p$
are neutron and proton, respectively.
And neutrino emission, via the reverse reactions, also changes composition.
As $Y_e$ changes, the thermodynamic properties of the fluid change.
\todo{add discussion of first law}

If the fluid is hot, both neutrino and antineutrino luminosities are high,
and near the emission surface, they will annihilate at high enough efficiencies
(via $\nu\,\bar{\nu}\rightarrow e^{-}\,e^{+}$) to create a lepton pair plasma.
Neutrino annihilation heating is one process by which accretion may
power an ultrarelativistic jet, forming a gamma ray burst, as described in
Sec.~\ref{ssec:how_common}, and explored in our model in
Sec.~\ref{sec:q_this_case}.
\todo{cite early Piran paper}

\subsubsection{Neutrino Oscillations}
\label{sssc:neutrino_osc}
Also, when neutrino number densities are high,
\todo{necessary in vacuum oscillations?}
neutrinos can interact in a more subtle way: converting from one flavor into
another, by a process called neutrino oscillation.
(The Standard Model of particle physics predicts three neutrino flavors,
corresponding to the three generations of leptons: electrons, muons, and tauons.
\todo{`generations'?}
Each neutrino has a corresponding antineutrino, yielding six distinct neutrino
species: $\nu_e$, $\bar{\nu}_e$, $\nu_\mu$, $\bar{\nu}_\mu$, $\nu_\tau$,
and $\bar{\nu}_\tau$.)
\todo{mass$\neq$flavor eigenstates}
Neutrino flavor oscillations were first proposed in 197?
\todo{find/cite}
and were later found to explain the long-standing solar physics problem in which
neutrino observatories measured an otherwise unexplainable deart of $\nu_?$-type
\todo{which $\nu$?}
neutrinos from the Sun. Further theoretical explorations in the context of
supernovae have revealed that
\todo{cite}
the presence of matter can significantly amplify the oscillation likelihood.
Another amplification of the electron-type oscillation can arise if the
$\bar{\nu}_e$ self-interaction potential is greater than the $\nu_e$ potential.
\todo{cite Malkus+2014}
(The self-interaction potential at any given point in space is a function of
neutrino energies, number densities, and the angular spread of incoming
neutrinos.) All of these conditions may obtain in and around a post-merger
nuclear accretion disk. Collective oscillations would change the neutrino
signal detected from a nearby merger, and could suppress or amplify neutrino
interactions with nearby matter, for example the nucleosynthesis processes
described below.

\subsubsection{Nucleosynthesis}
\label{sssc:nucleosynthesis}
The eventual fate of matter in a neutron star--black hole merger is to drain
across the one-way boundary of the horizon. From then on, its only effect on
the universe outside the black hole is a gravitational effect.
However, in some mergers, depending on the intrinsic sizes and spins of the
two bodies, and on the eccentricity of their final orbits, some fraction of the
neutron star matter may be ejected and escape, as mentioned in
Sec.~\ref{sec:how_common} \citep{latt1974-bhns_ejecta}.
Matter may become gravitationally unbound from the central bodies by dynamical
ejection or by radiation pressure, or by plasma interactions with magnetic
fields.
Such matter, originally free baryons and leptons, will go through a change of
composition as it leaves the merger environment, expanding and cooling.
Heavier nuclei will form from its free nucleon and light-nuclei constituents.
This process, called nucleosynthesis, can be dramatically modified by neutrino
irradiation, both heating up the escaping fluid, and changing its composition.
For example, with simplifying assumptions for their accretion disk model,
Surman et al.\ 2014
\todo{fix cite}
showed in one case, that $\nu_e$ radiation dominated over $\bar{\nu}_e$
unexpectedly, causing large amounts of $^{}56$Ni to form, instead of the
heavier lanthanides expected when $\bar{\nu}_e$ dominates.

\subsubsection{Direct Detection}
\label{sssc:nu_detection}
Finally, neutrinos emitted during merger and accretion could be detected by
existing and future neutrino observatories. Neutrinos, as we showed earlier,
carry away most of the thermal energy generated in a neutron star--neutron star
\nsns or neutron star--black hole \nsbh merger, because photons are trapped in
the dense matter. And neutrinos are produced copiously in nuclear matter at
temperatures above $\sym1$~MeV.
\todo{check/cite}
Neutrino luminosities are of the order $10^{53}\,{\rm erg\,s}^{-1}$. Even
though neutrinos interact very weakly with matter, detectors are in operation
that could detect a neutrino signal this strong from a merger in the
Milky Way Galaxy.
\todo{cite Caballero/McGlaughlin 2009}.

\section{How Long Does it Take?}
\label{sec:timescales}
The main timescale of a neutron star--black hole merger is the dynamical
timescale. This is the time over which gravity causes changes to the fluid.
As we will see, it is essentially unchanging throughout the merger.
This is an astonishing fact, since at the beginning, the fluid is essentially
static in the neutron star, and at the end it is highly dynamical, in the disk.
The dynamical timescales associated with these two fluid configurations are:
\begin{align}
  T_{\rm dyn,star} &\approx (G\rho)^{-1/2} \nonumber \\
  T_{\rm dyn,disk} &\approx 2\pi/\Omega. \nonumber
\end{align}
where $G$ is Newton's gravitational constant, $\rho$ the star's density,
and $\Omega$ the angular frequency of the disk.
Before merger, the neutron star's central density is
$\rho\sim10^{14}\,{\rm g\,cm}^{-3}$---thus $T_{\rm dyn,star}\approx0.5$~ms.
After merger, the bulk of the disk orbits the black hole at a distance of
a few horizon radii, where the Kepplerian orbital angular frequency for a black
hole of a few solar masses is
$\Omega\sim1000\,{\rm rad\,s^{-1}}$---thus $T_{\rm dyn,disk}\approx0.5$~ms.

(As an aside, we can see the correspondence between these two dynamical
timescales by using Keppler's law, $GM=\Omega^2 R^3$, from
Sec.~\ref{sec:neutrino_roles},
to write $\Omega^{-1}=(R^3/GM)^{1/2}=(G\rho)^{-1/2}$.
In this sense, the fact that $T_{\rm dyn}$ doesn't change over the different
epochs of the merger is not so surprising, since the fundamental gravitational
scales of the problem, mass and length, are basically of order $1\,M_\odot$
and $10$~km throughout.)

So the main timescale of a neutron star--black hole merger is of order
$T\sim1$~ms.
For a physical description of the changes occuring in a complex system---like
the one at hand, involving gravity, electromagnetism, nuclear interactions, and
radiation---timescales guide us in making powerful simplifications.
If a particular subset of physics has a timescale
$T_{\rm slow\, process}\gg T$, we can ignore it:
its related physical quantities may be fixed at their initial level.
If some other subset of physics has a timescale
$T_{\rm fast\, process}\ll T$, we can also ignore it:
its related physical quantities may be averaged over a cycle (if the process
is periodic) or set at some quasi-equilibrium value (if the process is secular).

\subsubsection{Gravitational Radiation}
The gravitational radiation timescale is how long it takes for gravitational
waves to change the system. We may also think of it as the time till merger.
The relativistic dynamics of the merger
are nonlinear, and when the binary is in close orbit, this timescale can only
be computed numerically. However, at large separations, when the bodies are
moving much slower than the speed of light, $v/c\ll1$, a linear approximation
\todo{PN? what order?}
can be made (see for example, \citealt[Sec.~16.4]{shap1983-bh_wd_ns}):
\begin{equation}
  T_{\rm grav} \approx \frac{5}{256}\frac{c^5}{G^3}\frac{R^4}{M^2\mu},
\end{equation}
where $M=m_1+m_2$, and $\mu \equiv m_1 m_2/(m_1+m_2)$, and $a$ is the semi-major
axis of the orbit, $R$ for extreme mass ratio binaries, and $2R$ for equal mass
binaries.
\todo{is that right?}
Even when the binary is in close orbit, this approximation is correct to order
of magnitude.
\todo{calculate for our system}

Gravitational radiation is, therefore, an important process to consider in the
merger.

\subsubsection{Particle Interactions}
Strong nuclear forces act on a timescale of $10^{-21}\,{\rm s}$ at the high
\todo{check/cite}
densities of the neutron star, and the high temperatures and densities of the
merger. This is the timescale over which
nuclei form and dissociate. It is so small compared to the main timescale of
of the merger, that we may consider the nuclear reactions to be in equilibrium.
This is the nuclear statistical equilibrium assumption, by which the fluid's
composition of relative particle species may be entirely parameterized by a
single parameter, the electron fraction $Y_e$.

But the weak force is much...weaker.
The weak timescale, or how long it takes $Y_e$ to change due to the charged
current reactions (Eqns.~\ref{eqn:beta_n_to_p} and~\ref{eqn:beta_p_to_n} and
their reverse processes) is
\begin{equation}
  T_{\rm weak} \approx \frac{M}{m_U} \frac{\langle Y_e \rangle}{R_\nu},
\end{equation}
where $M$ is the total fluid mass, $m_U$ the average nucleonic mass,
$\langle Y_e \rangle$ the average electron fraction, and
$R_\nu \equiv R_{\nu_e}-R_{\bar{\nu}_e}$ the net lepton number carried
away by neutrinos.
Simulations, like the one presented in this thesis (see
Fig.~\ref{fig:neutrinos_by_species}, have shown that
$R_\nu\gtrsim10^{57}\,{\rm s}^{-1}$.
For a fraction of a solar mass of material, $M/m_U\sim10^{56}$~baryons.
The electron fraction of a neutron star is very low, and we may estimate
$\langle Y_e \rangle\sim0.1$. Thus $T_{\rm weak}\sim10$~ms. Of course, to
compute this estimate, we assume the disk changes homogeneously; in fact,
some regions radiate neutrinos more luminously than others, and the weak
timescale in more luminous regions is much shorter.
Weak processes, then, are important to consider in the merger.

%\subsubsection{Radiation}
%The radiation timescale is the time over which the radiation field responds
%to changes in the matter. If the matter is transparent, that is the neutrino
%mean free path $\lambda_\nu$ is larger than the length scale of fluid variations
%$L$, neutrinos emitted from one side travel to the other side in
%a light-crossing time,
%\begin{equation}
%  T_{\rm rad,\, free} \approx \frac{L}{c}.
%\end{equation}
%If, however, the matter is opaque, that is the neutrino mean free path is
%smaller than fluid length scales, neutrinos emitted from deep within the fluid
%experience many scatterings before escaping to the outer reaches. The radiation
%field changes on a diffusion timescale,
%\begin{equation}
%  T_{\rm rad,\,diff} \approx \frac{\lambda_\nu}{c}.
%\end{equation}

\section{What's the History Behind this Study?}
\label{sec:history}
As we found in the previous sections, neutrinos are interesting and important
players in neutron star--black hole mergers.
What has the astrophysics community already learned about them
in this context?\footnote{
some of this section is reproduced from \citealt[Sec.~1]{deat2013-leakage}}

Because the accretion disk spans neutrino-opaque and neutrino-transparent regimes
(see Sec.~\ref{sec:neutrino_roles}), a full six-dimensional evolution of the
neutrino fields (1 energetic + 3 spatial + 2 angular dimensions) is needed
to completely describe the coupling between radiation and matter.
This is impossible with current computational resources. Fortunately, many of
the essential features of the radiation can be captured using a simple
leakage scheme. Rather than performing actual radiation transport, leakage
schemes remove energy and alter lepton number at rates based on the local
free-emission and diffusion rates. Leakage certainly neglects some arguably
important effects (for example, neutrino-driven winds), but it captures
the basic energetics and composition drift of the post-merger system.
For an analysis of leakage in core-collapse supernova simulations,
see \citet{ott2013-cc_leakage}.
\todo{say: we focus our review on leakage sims...}

Neutrino leakage simulations have been used to build models of neutron
star--neutron star \nsns mergers for nearly two decades
\citep{
ruff1996-leakage_part1,
ross2002-leakage_part1}.
\citet{jank1999-leakage_bhns} and \citet{ross2004-leakage_bhns} developed models
of neutron star--black hole \nsbh mergers using Newtonian gravity, with
phenomenological treatments of relativistic effects like the back reaction of
gravitational waves on the fluid, and the orbital instability at radii close to
the black hole.
These models predict neutrino luminosities of $10^{51\text{--}53}\rm\,erg\,s^{-1}$.

Leakage techniques have recently been incorporated into general relativistic
simulations of stellar core collapse
\citep{seki2011-cc_leakage, ott2013-cc_leakage}
and neutron star--neutron star \nsns mergers
\citep{seki2011-nsns_leakage,kiuc2012-nsns_leakage_hyperons}.
But the first relativistic neutron star--black hole simulations with a
nuclear treatment of the fluid and neutrino feedback were presented in
\cite{deat2013-leakage}, reproduced in Chap.~\ref{chap:leakage}.

\section{Why this Particular Model?}
\label{sec:why_this_model}
In this thesis, I focus our attention on neutron star--black hole mergers, and in
particular, on one model of such a merger, with a specific set of initial
parameters. In this section, I explain that narrow focus.

Neutron star--neutron star \nsns and neutron star--black hole \nsbh mergers are
similar in many respects. They both evolve through an epoch dominated by
gravitational wave emission: inspiral, during which their orbits rapidly decay.
In both types of mergers, the neutron stars experience strong tidal forces,
distorting them into oblate spheres, and finally disrupting them.
And except for very low mass neutron star--neutron star \nsns binaries, with
total masses not exceeding the maximum neutron star mass, $\sym2\,M_\odot$,
all instances of both types of merger end up finally in the same state, a
single black hole with not matter nearby.

But the epoch between neutron star disruption and final black hole quiescence
may be quite distinct between these two classes of mergers. Generally, neutron
star--neutron star \nsns mergers yield a rapidly spinning blob of more than
$2\,M_\odot$ of matter, a supramassive neutron star,
\todo{super or supra?}
supported against collapse, briefly, by centrifugal effects.
On the other hand, neutron star--black hole \nsbh mergers transition directly
to the disk stage, or else the neutron star is swallowed whole immediately upon
disrupting, and no disk forms.

In this work I focus on neutron star--black hole mergers \nsbh, where the
accretion epoch plays a central role. Furthermore, though both classes of
merger can disperse nuclear matter into the interstellar medium via winds and
dynamical ejection, the latter effect can be much more significant
in neutron star--black hole
mergers because of the greater compactness, and therefore stronger
tidal forces due to the black hole. In this and future work, I am especially
interested in this dynamical ejecta, its role in galactic nucleosynthesis,
and its potential to emit a detectable flash of light. Also, though by most
\todo{cite low bhbh rate predictions}
predictions neutron star--black hole mergers are the least likely compact
binary merger in the nearby universe (see Sec.~\ref{ssec:how_common}),
they are more easily detectable by LIGO and Virgo. This is because the
strength of the peak gravitational wave signal scales with the total mass,
\todo{check}
and the average stellar mass black hole is likely many times as massive as the
average neutron star ($10\,M_\odot$ vs.\ $2\,M_\odot$).
\todo{cite mass predictions}
\todo{this last bit is weak, needs more nuance: e.g.\ frequency bucket...}

In this work I present one model of a neutron star--black hole merger,
with a high black hole dimensionless spin of
$a^* \equiv J_{\rm BH}c/GM_{\rm BH}^2 =0.9$ aligned with the binary's orbital
angular momentum,
component masses of $M_{\rm BH}=5.6\,M_\odot$ and
$M_{\rm NS}=1.4\,M_\odot$,
and a neutron star radius of 12.7~km,
so that the mass ratio is $q \equiv M_{\rm BH}/M_{\rm NS}=4$,
and the neutron star compactness is
$\mathcal{C}_{\rm NS} \equiv G M_{\rm NS}/R_{\rm NS}c^2=0.16$.
We are particularly interested in these binary parameters because such a system
yields a massive accretion disk.

Heuristically it makes sense that $a*$, $q$, $M_{\rm BH}$, and
$\mathcal{C}_{\rm NS}$ have an effect on the amount of matter remaining outside
the black hole after tidal forces cause the neutron star to disrupt.
Tidal forces behave approximately as $M_{\rm BH} R_{\rm NS}/r^3$,
which can be evaluated at the radius of the last stable circular orbit,
inside of which the neutron star plunges, 
\begin{equation}
  \label{eqn:tidal_forces}
  F_{\rm tidal} \propto
  \left(q \,\mathcal{C}_{\rm NS}\, M_{\rm BH} \,f(a^*)^3\right)^{-1},
\end{equation}
where $f(a^*) \equiv M_{\rm BH}/r_{\rm ISCO}(a^*)$ decreases monotonically with increasing
prograde spin, and $r_{\rm ISCO}$ is the radius of the last (a.k.a\ inner-most)
stable circular orbit. Even this simple relationship indicates that
tidal disruption forces at the last stable circular orbit are greater for smaller
$q$, smaller $C_{\rm NS}$, smaller $M_{\rm BH}$, and larger $a^*$.
Simulations across the parameter space confirm these trends.
\todo{cite}

In fact, these four parameters are not independent. Since the compactness of a
cold neutron star is completely specified by its mass\footnote{we are
assuming the equation of state of neutron star matter is known}, we can write
Eqn.~\ref{eqn:tidal_forces} as
\begin{equation}
  F_{\rm tidal} \propto
  \left(q^2 \,\mathcal{C}_{\rm NS}(M_{\rm NS})\, M_{\rm NS}\, f(a^*)^3\right)^{-1},
\end{equation}
which describes a three-dimensional parameter space of possible mergers:
$\{q,M_{\rm NS},a^*\}$.
Furthermore, these parameters are constrained by observed populations. We
choose to study $M_{\rm NS}=1.4\,M_{\odot}$, a canonical neutron star mass, near
the center of the observed masses.
\todo{check/cite stellarcollapse.org}
This choice fixes the compactness of the star, and leaves $q$ and $a^*$
independently specifiable.
Field populations of black holes are observed to lie in the mass range ?--?,
and spin range ?--?.
\todo{fill in/cite Ott+ 2014 bayesian analysis SN endstates, or \"Ozel}
To maximize disk mass we choose a black hole mass toward the bottom of this
range, $M_{\rm BH}=5.6\,M_\odot$, and a black hole spin toward the top of this
range, $a^*=0.9$.

Using phenomenological models fit to numerical simulations,
\cite{fouc2012-disk_mass} and \cite{pann2010-disk_mass} independently published
disk mass predictions parameterized by $\{q,M_{\rm NS},a^*\}$.
Using Foucart's model, we predict a disk mass of
$M_{\rm disk,predict}\sim ?\,M_\odot$.
\todo{calculate, fill in}.

\section{What's the Scope and Trajectory of this Thesis?}
\label{sec:scope}

At the heart of this thesis, I present a model of a neutron star--black hole merger,
accounting in an approximate way for the major effects of neutrinos on the
post-merger accretion disk (Chap.~\ref{chap:leakage}).
This model then provides a starting point from which to examine the properties of
the radiation, in particular the spatial, energy, and angular distribution of the
escaping neutrinos
(Chap.~\ref{chap:ray_tracing}, Secs.~\ref{sec:f_algorithm}--\ref{sec:f_this_case}).
Finally, I use the neutrino distributions inferred from the model to investigate
the energetics of a very important astrophysical process: the formation of a
neutrino-driven jet which could power a short gamma ray burst
(Chap.~\ref{chap:ray_tracing}, Secs.~\ref{sec:q_algorithm}--\ref{sec:q_this_case}).

In addition to a reference section on conventions (Sec.~\ref{sec:conventions}),
Chap.~\ref{chap:intro} provides a technical introduction to the two physical
frameworks for the model presented here:
fluid dynamics (Sec.~\ref{sec:gr_hydro}) and
radiation transport (Sec.~\ref{sec:rad_transport})
in strong gravity.

I have aimed the preceding chapter in a general direction, toward students from
other fields than physics; the remainder is aimed squarely at physicists.
