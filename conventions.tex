% Notation Conventions
% Brett Deaton -- Spring 2015

\section{Conventions}
\label{sec:conventions}

I adopt the standard general relativists convention of a metric signature
$(-,+,+,+)$.
Unless otherwise noted, I display equations in natural geometric units, so that
Newton's gravitational constant, $G$, and the speed of light, $c$, have unit
value.

I use abstract tensor notation in general: a tensor typeset with its indeces
represents the geometric object in all of its coordinate freedom. In other words
$T^{\alpha \beta}$ is identical to $T$, and is not $T$'s $\alpha\beta$-th
component.
In some equations, I use coordinate index notation accompanied with a warning.
Spatial tensors are given indices from the Latin alphabet ($\alpha,\beta...$);
they exist on a 3-dimensional manifold. Spacetime tensors are given indices from
the Greek alphabet ($a,b...$); they exist on a 4-dimensional manifold.
In instances of coordinate index notation, it follows naturally that
$\alpha,\beta...\in\{1,2,3\}$ and $a,b...\in\{0,1,2,3\}$.

Of course some symbols have labels other than their tensor indeces.
In particular, I use subscript labels before tensor indeces (e.g.\ the
momentum of a neutrino is $p_{\nu \alpha}$). I allow context to clarify when
an index ranges over some set other than dimension (e.g.\ the momentum of the
i-th flavor of neutrino is $p_{\nu_i \alpha}$). The subscript $\nu$ is always
used as a label, not as a tensor index.

\begin{table}
  \centering
  \begin{tabular}{rll}
    \textbf{Symbol}       & \textbf{Object}           & \\%\textbf{Additional Notes} \\
    $\langle m_U \rangle$ & average nucleon mass      & \\
    $n_b$                 & baryon number density     & \\
    $\rho$                & rest density              & $\rho=\langle m_U \rangle n_b $ \\
    $\epsilon$            & specific internal energy  & not including the mass energy \\
    $h$                   & specific enthalpy         & $h=1+\epsilon+P/\rho$\\
    $P$                   & pressure                  & \\
    $T$                   & temperature               & \\
    $T^{\alpha \beta}$    & stress-energy tensor      & \\
    $Y_e$                 & electron fraction         & $Y_e=n_e/n_b$\\
    $\rho_*$              & density evolution variable& $\rho_*=\sqrt{g}W\rho$ \\
    $\tilde\tau$          & energy evolution variable & $\tilde\tau=\rho_*(hW-1)-\sqrt{g}P$ \\
    $\tilde S$            & momentum evolution variable & $\tilde S=\rho_*hu_i$\\
    $v^i$                 & Lagrangian velocity       & a.k.a.\ transport velocity, $v^i=u^i/u^0$ \\
    $u^\alpha$            & four-velocity             & $u^a=dx^a/d\tau$ \\
    $d\tau$               & spacetime interval        & $d\tau=-dx^a dx^b \psi_{ab}$ \\
    $W$                   & Lorentz factor            & $\alpha u^0$ (if no gravity $W=1/\sqrt{1-v^2}$) \\
    $\tau$                & optical depth             & \\
    $Q_\nu$               & local energy emission rate& \\
    $R_\nu$               & local lepton number rate  & \\
    $\mu$                 & chemical potential        & $\mu=T\eta$ \\
    $p_\alpha$            & momentum 1-form           & \\
    $\varepsilon$         & neutrino energy           & used for asymptotic and local \\
    $\psi_{\alpha\beta}$  & spacetime metric          & \\
    $g_{ij}$              & spatial metric            & \\
    $\beta^i$             & shift                     & where $\beta_i = \beta^j g_{ij}$ \\
    $\alpha$              & lapse                     & \\
  \end{tabular}
  \caption[Symbols used in the text]{
    Some symbols used in this text.
    Indices are used in the ``Symbol'' column in an abstract sense, and in the
    ``Additional Notes'' column in a component sense.
  }
  \label{tab:conventions}
\end{table}
