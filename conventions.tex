% Notation Conventions
% Brett Deaton -- Spring 2015

\section{Conventions}
\label{sec:conventions}

I adopt the standard general relativists metric signature convention
$(-,+,+,+)$.
In Chap.~\ref{chap:leakage}, I use natural geometric units
in which Newton's gravitational constant $G$, and the speed of light $c$, have unit
value; In the other chapters, I make $G$ and $c$ explicit.

I use abstract tensor notation in general: a tensor typeset with its indices
represents the geometric object in all of its coordinate freedom. In other words
$T^{\alpha \beta}$ is identical to $T$, and is not $T$'s $\alpha\beta$-th
component.
In some equations, I use coordinate index notation accompanied with a warning.
Spatial tensors are given indices from the Latin alphabet ($a,b...$);
they exist on a three-dimensional manifold. Spacetime tensors are given indices from
the Greek alphabet ($\alpha,\beta...$); they exist on a four-dimensional manifold.
In instances of coordinate index notation, it follows naturally that
$a,b...\in\{1,2,3\}$ and $\alpha,\beta...\in\{0,1,2,3\}$.

Of course some symbols have labels other than their tensor indices.
In particular, I use subscript labels before tensor indeces (e.g.\ the
momentum of a neutrino is $p_{\nu \alpha}$). I allow context to clarify when
an index ranges over some set other than dimension (e.g.\ the momentum of the
i-th flavor of neutrino is $p_{\nu_i \alpha}$). The subscript $\nu$ is always
used as a label, not as a tensor index.

Repeated up and down indices imply a contraction. For example
in cartesian representation, $\beta_i\beta^i$ is computed component-wise by
$\beta_x\beta^x+\beta_y\beta^y+\beta_z\beta^z$.

\begin{table}
  \centering
  \begin{tabular}{rll}
    \textbf{Symbol}       & \textbf{Object}           & \\%\textbf{Additional Notes} \\
    $n_b$                 & baryon number density     & \\
    $n_e$                 & electron number density   & \\
    $n_i$                 & ion number density        & $n_i=n_{e^+}+n_{e^-}+n_p=2n_b Y_e$\\
    $Y_e$                 & electron fraction         & $Y_e=n_e/n_b$\\
    $\rho$                & rest density              & $\rho \equiv m_U n_b $ \\
    $\epsilon$            & specific internal energy  & not including the mass energy \\
    $h$                   & specific enthalpy         & $h=1+\epsilon+P/\rho$\\
    $P$                   & pressure                  & \\
    $T$                   & temperature               & \\
    $T^{\alpha \beta}$    & stress-energy tensor      & \\
    $\rho_*$              & density evolution variable& $\rho_*=\sqrt{g}W\rho$ \\
    $\tilde\tau$          & energy evolution variable & $\tilde\tau=\rho_*(hW-1)-\sqrt{g}P$ \\
    $\tilde S_i$          & momentum evolution variable & $\tilde S_i=\rho_*hu_i$\\
    $v^i$                 & Lagrangian velocity       & a.k.a.\ transport velocity, $v^i=u^i/u^t$ \\
    $u^\alpha$            & fluid four-velocity       & $u^a=dx^a/d\tau$ \\
    $U^\alpha$            & observer four-velocity    & used if we want to distinguish from $u^\alpha$ \\
    $d\tau$               & spacetime interval        & $d\tau=-dx^a dx^b \psi_{ab}$ \\
    $W$                   & Lorentz factor            & $\alpha u^t$ (if no gravity $W=1/\sqrt{1-v^2}$) \\
    $\tau$                & optical depth             & \\
    $Q_\nu$               & local energy emission rate& \\
    $R_\nu$               & local lepton number rate  & \\
    $\mu$                 & chemical potential        & we also use $\eta\equiv\mu/T$ \\
    $p_\alpha$            & momentum 1-form           & \\
    $\varepsilon$         & neutrino energy           & used for asymptotic and local energy\\
    $\psi_{\alpha\beta}$  & spacetime metric          & \\
    $g_{ij}$              & spatial metric            & \\
    $\beta^i$             & shift                     & where $\beta_i = \beta^j g_{ij}$ \\
    $\alpha$              & lapse                     & \\
    \nsns                 & double neutron star binary & \\
    \bhbh                 & double black hole binary  & \\
    \nsbh                 & neutron star--black hole binary &
  \end{tabular}
  \caption[Symbols used in the text]{
    Some symbols used in this text.
  }
  \label{tab:conventions}
\end{table}

\begin{table}
  \centering
  \begin{tabular}{rlll}
    \textbf{Symbol} & \textbf{Value}       & \textbf{Units}           & \textbf{Description} \\
    $G$             & $6.67\times10^{-8}$  & cm$^3$ g$^{-1}$ s$^{-2}$ & Newton's gravitational constant \\
    $c$             & $3.00\times10^{10}$  & cm s$^{-1}$              & speed of light in vacuum \\
    $k_{\rm B}$     & $8.62\times10^{-11}$ & MeV K$^{-1}$             & Boltzmann's constant \\
    $G_{\rm F}$     & $2.3\times10^{-22}$  & cm MeV$^{-1}$            & Fermi coupling constant \\
    $G_{\rm F}/(\hbar c)^3$ & $1.17\times10^{-11}$ & MeV$^{-2}$       & ", naturalized units, $\hbar=c=1$ \\
    $\sin^2\theta_w$& 0.231                &                          & weak-mixing angle \\
    $m_e$           & 0.511                & MeV c$^{-2}$             & mass of electron \\
    $m_U$           & 939                  & MeV c$^{-2}$             & average nucleon mass, $(m_n+m_p)/2$ \\
    $m_n-m_p$       & 1.29                 & MeV c$^{-2}$             & mass difference of $n$ and $p$ \\
    $M_\odot$       & $2.00\times10^{33}$  & g                        & solar mass \\
    $\hbar$         & $6.58\times10^{-22}$ & MeV s                    & reduced Planck's constant, $h/2\pi$ \\
  \end{tabular}
  \caption[Physical constants used in the text]{
    Some physical constants used in this text. Taken from the Particle Data
    Group, Particle Physics Booklet \citep{oliv2014-pdg}.
    I am only interested in scales, so I use three significant figures here.
    Note, the conversion factor between $G_{\rm F}$ in cgs and
    naturalized units is the naturalized length scale,
    $\hbar c/\varepsilon=1.97\times10^{-11}/(\varepsilon/{\rm MeV})$~cm.
  }
  \label{tab:constants}
\end{table}

\subsection{Foliating Spacetime}
\label{ssec:adm_metric}
In everything that follows we employ the 3+1 splitting of the spacetime metric
\citep[Chap.~7]{witt1962-book_with_adm_intro}\footnote{this book is now out of
print, but the relevant chapter is available on the arXiv \citep{arno2008-adm}.},
\begin{equation}
  -d\tau^2 \equiv \alpha^2dt^2 + g_{ij}(dx^i+\beta^idt)(dx^j+\beta^jdt).
\end{equation}
The cartesian decomposition of the metric and its inverse is
\begin{equation}
  \psi_{\mu\gamma} :=
  \left(
  \begin{matrix}
    -\alpha^2 + \beta^i \beta_i  & \beta_i \\
    \beta_j                      & g_{ij}
  \end{matrix}
  \right)
  \qquad
  \psi^{\mu\gamma} :=
  \left(
  \begin{matrix}
    -\frac{1}{\alpha^2}          & \frac{\beta^j}{\alpha^2} \\
    \frac{\beta^i}{\alpha^2}     & g^{ij} - \frac{\beta^i \beta^j}{\alpha^2}
  \end{matrix}
  \right),
\end{equation}
where $\beta_i=\beta^j g_{ij}$, and $g^{ij}g_{ik}=\delta^{j}_{k}$.
This decomposition generates a foliation of the spacetime manifold into
three-dimensional spacelike submanifolds, whose coordinates are related
by the timelike congruence $n^\mu=dx^\mu/\alpha dt$, or in cartesian
decomposition, $n^\mu:=(1/\alpha,-\beta^i/\alpha)$.

