\chapter{Introduction}
\label{chap:intro}

% Notation Conventions
% Brett Deaton -- Spring 2015

\section{Conventions}
\label{sec:conventions}

I adopt the standard general relativists metric signature convention
$(-,+,+,+)$.
In Chap.~\ref{chap:leakage}, I use natural geometric units
in which Newton's gravitational constant $G$, and the speed of light $c$, have unit
value; In the other chapters, I make $G$ and $c$ explicit.

I use abstract tensor notation in general: a tensor typeset with its indeces
represents the geometric object in all of its coordinate freedom. In other words
$T^{\alpha \beta}$ is identical to $T$, and is not $T$'s $\alpha\beta$-th
component.
In some equations, I use coordinate index notation accompanied with a warning.
Spatial tensors are given indices from the Latin alphabet ($a,b...$);
they exist on a 3-dimensional manifold. Spacetime tensors are given indices from
the Greek alphabet ($\alpha,\beta...$); they exist on a 4-dimensional manifold.
In instances of coordinate index notation, it follows naturally that
$\alpha,\beta...\in\{1,2,3\}$ and $a,b...\in\{0,1,2,3\}$.

Of course some symbols have labels other than their tensor indeces.
In particular, I use subscript labels before tensor indeces (e.g.\ the
momentum of a neutrino is $p_{\nu \alpha}$). I allow context to clarify when
an index ranges over some set other than dimension (e.g.\ the momentum of the
i-th flavor of neutrino is $p_{\nu_i \alpha}$). The subscript $\nu$ is always
used as a label, not as a tensor index.

\begin{table}
  \centering
  \begin{tabular}{rll}
    \textbf{Symbol}       & \textbf{Object}           & \\%\textbf{Additional Notes} \\
    $n_b$                 & baryon number density     & \\
    $\rho$                & rest density              & $\rho \equiv m_U n_b $ \\
    $\epsilon$            & specific internal energy  & not including the mass energy \\
    $h$                   & specific enthalpy         & $h=1+\epsilon+P/\rho$\\
    $P$                   & pressure                  & \\
    $T$                   & temperature               & \\
    $T^{\alpha \beta}$    & stress-energy tensor      & \\
    $Y_e$                 & electron fraction         & $Y_e=n_e/n_b$\\
    $\rho_*$              & density evolution variable& $\rho_*=\sqrt{g}W\rho$ \\
    $\tilde\tau$          & energy evolution variable & $\tilde\tau=\rho_*(hW-1)-\sqrt{g}P$ \\
    $\tilde S$            & momentum evolution variable & $\tilde S=\rho_*hu_i$\\
    $v^i$                 & Lagrangian velocity       & a.k.a.\ transport velocity, $v^i=u^i/u^t$ \\
    $u^\alpha$            & fluid four-velocity       & $u^a=dx^a/d\tau$ \\
    $U^\alpha$            & observer four-velocity    & used if we want to distinguish from $u^\alpha$ \\
    $d\tau$               & spacetime interval        & $d\tau=-dx^a dx^b \psi_{ab}$ \\
    $W$                   & Lorentz factor            & $\alpha u^t$ (if no gravity $W=1/\sqrt{1-v^2}$) \\
    $\tau$                & optical depth             & \\
    $Q_\nu$               & local energy emission rate& \\
    $R_\nu$               & local lepton number rate  & \\
    $\mu$                 & chemical potential        & $\mu=T\eta$ \\
    $p_\alpha$            & momentum 1-form           & \\
    $\varepsilon$         & neutrino energy           & used for asymptotic and local \\
    $\psi_{\alpha\beta}$  & spacetime metric          & \\
    $g_{ij}$              & spatial metric            & \\
    $\beta^i$             & shift                     & where $\beta_i = \beta^j g_{ij}$ \\
    $\alpha$              & lapse                     & \\
  \end{tabular}
  \caption[Symbols used in the text]{
    Some symbols used in this text.
    Indices are used in the ``Symbol'' column in an abstract sense, and in the
    ``Additional Notes'' column in a component sense.
  }
  \label{tab:conventions}
\end{table}

\begin{table}
  \centering
  \begin{tabular}{rlll}
    \textbf{Symbol} & \textbf{Value}       & \textbf{Units}           & \textbf{Description} \\
    $G$             & $6.67\times10^{-8}$  & cm$^3$ g$^{-1}$ s$^{-2}$ & Newton's gravitational constant \\
    $c$             & $3.00\times10^{10}$  & cm s$^{-1}$              & speed of light in vacuum \\
    $k_{\rm B}$     & $8.62\times10^{-11}$ & MeV K$^{-1}$             & Boltzmann's constant \\
    $G_{\rm F}$     & $2.3\times10^{-22}$  & cm MeV$^{-1}$            & Fermi coupling constant \\
    $G_{\rm F}/(\hbar c)^3$ & $1.17\times10^{-11}$ & MeV$^{-2}$       & ", naturalized units, $\hbar=c=1$ \\
    $\sin^2\theta_w$& 0.231                &                          & weak-mixing angle \\
    $m_e$           & 0.511                & MeV c$^{-2}$             & mass of electron \\
    $m_U$           & 939                  & MeV c$^{-2}$             & average nucleon mass, $(m_n+m_p)/2$ \\
    $M_\odot$       & $2.00\times10^{33}$  & g                        & solar mass \\
    $\hbar$         & $6.58\times10^{-22}$ & MeV s                    & reduced Planck's constant, $h/2\pi$ \\
  \end{tabular}
  \caption[Physical constants used in the text]{
    Some physical constants used in this text. Taken from the Particle Data
    Group, Particle Physics Booklet \citep{oliv2014-pdg}.
    I am only interested in scales, so I use three significant figures here.
    Note, the conversion factor between $G_{\rm F}$ in cgs and
    naturalized units is the naturalized length scale,
    $\hbar c/\varepsilon=1.97\times10^{-11}/(\varepsilon/{\rm MeV})$~cm.
  }
  \label{tab:constants}
\end{table}


\section{General Relativistic Hydrodynamics}
\label{sec:gr_hydro}
Fluid dynamics are described by the hydrodynamics equations. In
Euclidean space, for a nonviscous fluid:
\begin{align}
  \label{eqn:hydro_flat_mass}
  \partial_t \rho + \nabla_i S^i       &= 0 \\
  \label{eqn:hydro_flat_mom}
  \partial_t S^j +  \nabla_i T^{ij}    &= f^j \rho \\
  \label{eqn:hydro_flat_ener}
  \partial_t (\frac{1}{2}\rho v^2+u) + \nabla_i(\frac{1}{2}\rho v^2+u+P)v^i &= \rho f_i v^i.
\end{align}
Here $\rho$ is mass density,
$S^i\equiv \rho v^i$ momentum density,
$T^{ij}\equiv \rho v^i v^j + P\delta^{ij}$ the stress tensor,
$f^j=f_j$ the sum of forces per unit mass acting on the fluid,
such as radiation pressure or gravity,
$u$ the specific internal energy, and
$P$ the pressure.
The total specific energy is $e=\rho v^2/2+u$.

This system of five equations (Eqn.~\ref{eqn:hydro_flat_mom} is a composition
of three equations) and six variables $\{\rho,u,P,v^i\}$ is closed by a sixth
equation, the equation of state, relating pressure to internal energy and
density, $P(u,\rho)$.
For an ideal monatomic gas,
we are more familiar with the parametric form of the equation
of state, the two equations $P(\rho,T)=\frac{3}{2m_U}\rho k_{\rm B} T$ and
$u(\rho,T)=\frac{1}{m_U}\rho k_{\rm B} T$.

The five dynamical equations
express for a fluid the conservation of
mass, Eqn.~\ref{eqn:hydro_flat_mass},
momentum, Eqn.~\ref{eqn:hydro_flat_mom}, and
energy, Eqn.~\ref{eqn:hydro_flat_ener}.
As such, they all take the form of hyperbolic conservation equations,
\begin{equation}
  \label{eqn:cons_hyper}
  \partial_t \mathcal{U} + \nabla_i \mathcal{F}^i = \mathcal{S},
\end{equation}
where $\mathcal{U}$ is the conserved quantity,
$\mathcal{F}^i$ its flux, and
$\mathcal{S}$ a source term pumping or draining the conserved quantity.

A hyperbolic equation is fully specified by its initial conditions. In this case
that means choosing an initial configuration $\vec{\mathcal{U}}(t=0)$,
or alternately, specifying the five primitive variables
$\rho(t=0)$, $T(t=0)$, and $v(t=0)$ throughout the physical domain.
Boundary conditions are required to define the behavior of the equations on
the domain boundaries.

A hyperbolic equation like Eqn.~\ref{eqn:cons_hyper}, may be solved via the
method of lines, computing the conservative fields across a spatial domain at
later times, $t>0$ by assuming the spatial dependence at $t=0$ is known. In
one dimension:
\begin{equation}
  \frac{\diff}{\diff t} \mathcal{U} = -\partial_i
\end{equation}

\subsubsection{Relativistic Minkowski}
But in fact, these equations fail when $v$ is close to $c$.
\todo{demonstrate?}
We may write them in their relativistic form:
\begin{align}
  \label{eqn:hydro_sr_mass}
  \nabla_\alpha \rho u^\alpha &= 0 \\
  \label{eqn:hydro_sr_mom}
  \nabla_\alpha T^{\alpha \beta} &= 0.
\end{align}
Here we have introduced $\nabla_\alpha:=(\frac{1}{c}\partial_t,\nabla_i)$,
$u^\alpha:=\gamma(c,v^i)$ the fluid 4-velocity,
$\gamma\equiv(1-v^2/c^2)^{-1/2}$ the Lorentz factor, and
$T^{\alpha\beta}$ the stress-energy tensor:
\begin{eqnarray}
  \label{eqn:se_tensor}
  T^{\alpha\beta} = T_{\rm fluid}^{\alpha\beta} + T_{\rm other}^{\alpha\beta} \\
  T_{\rm fluid}^{\alpha\beta} = (\rho+\frac{u+P}{c^2})u^\alpha u^\beta+P\eta^{\alpha\beta}, \\
\end{eqnarray}
where $\eta^{\alpha\beta}=\eta_{\alpha\beta}:={\rm diag}(-1,1,1,1)$ is the
Minkowski metric, and
$T_{\rm other}^{\alpha\beta}$ is the stress-energy tensor of other fields, like
the radiation field.
Eqns.~\ref{eqn:hydro_sr_mass} and~\ref{eqn:hydro_sr_mom} elegantly express the
conservation of rest mass and momentum-energy.
\todo{what are the conservatives here?}

\section{General Relativistic Radiation Transport}
\label{sec:rad_transport}

