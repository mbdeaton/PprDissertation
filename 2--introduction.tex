\chapter{Technical Introduction}
\label{chap:intro}

% Notation Conventions
% Brett Deaton -- Spring 2015

\section{Conventions}
\label{sec:conventions}

I adopt the standard general relativists metric signature convention
$(-,+,+,+)$.
In Chap.~\ref{chap:leakage}, I use natural geometric units
in which Newton's gravitational constant $G$, and the speed of light $c$, have unit
value; In the other chapters, I make $G$ and $c$ explicit.

I use abstract tensor notation in general: a tensor typeset with its indeces
represents the geometric object in all of its coordinate freedom. In other words
$T^{\alpha \beta}$ is identical to $T$, and is not $T$'s $\alpha\beta$-th
component.
In some equations, I use coordinate index notation accompanied with a warning.
Spatial tensors are given indices from the Latin alphabet ($a,b...$);
they exist on a 3-dimensional manifold. Spacetime tensors are given indices from
the Greek alphabet ($\alpha,\beta...$); they exist on a 4-dimensional manifold.
In instances of coordinate index notation, it follows naturally that
$\alpha,\beta...\in\{1,2,3\}$ and $a,b...\in\{0,1,2,3\}$.

Of course some symbols have labels other than their tensor indeces.
In particular, I use subscript labels before tensor indeces (e.g.\ the
momentum of a neutrino is $p_{\nu \alpha}$). I allow context to clarify when
an index ranges over some set other than dimension (e.g.\ the momentum of the
i-th flavor of neutrino is $p_{\nu_i \alpha}$). The subscript $\nu$ is always
used as a label, not as a tensor index.

\begin{table}
  \centering
  \begin{tabular}{rll}
    \textbf{Symbol}       & \textbf{Object}           & \\%\textbf{Additional Notes} \\
    $n_b$                 & baryon number density     & \\
    $\rho$                & rest density              & $\rho \equiv m_U n_b $ \\
    $\epsilon$            & specific internal energy  & not including the mass energy \\
    $h$                   & specific enthalpy         & $h=1+\epsilon+P/\rho$\\
    $P$                   & pressure                  & \\
    $T$                   & temperature               & \\
    $T^{\alpha \beta}$    & stress-energy tensor      & \\
    $Y_e$                 & electron fraction         & $Y_e=n_e/n_b$\\
    $\rho_*$              & density evolution variable& $\rho_*=\sqrt{g}W\rho$ \\
    $\tilde\tau$          & energy evolution variable & $\tilde\tau=\rho_*(hW-1)-\sqrt{g}P$ \\
    $\tilde S$            & momentum evolution variable & $\tilde S=\rho_*hu_i$\\
    $v^i$                 & Lagrangian velocity       & a.k.a.\ transport velocity, $v^i=u^i/u^t$ \\
    $u^\alpha$            & fluid four-velocity       & $u^a=dx^a/d\tau$ \\
    $U^\alpha$            & observer four-velocity    & used if we want to distinguish from $u^\alpha$ \\
    $d\tau$               & spacetime interval        & $d\tau=-dx^a dx^b \psi_{ab}$ \\
    $W$                   & Lorentz factor            & $\alpha u^t$ (if no gravity $W=1/\sqrt{1-v^2}$) \\
    $\tau$                & optical depth             & \\
    $Q_\nu$               & local energy emission rate& \\
    $R_\nu$               & local lepton number rate  & \\
    $\mu$                 & chemical potential        & $\mu=T\eta$ \\
    $p_\alpha$            & momentum 1-form           & \\
    $\varepsilon$         & neutrino energy           & used for asymptotic and local \\
    $\psi_{\alpha\beta}$  & spacetime metric          & \\
    $g_{ij}$              & spatial metric            & \\
    $\beta^i$             & shift                     & where $\beta_i = \beta^j g_{ij}$ \\
    $\alpha$              & lapse                     & \\
  \end{tabular}
  \caption[Symbols used in the text]{
    Some symbols used in this text.
    Indices are used in the ``Symbol'' column in an abstract sense, and in the
    ``Additional Notes'' column in a component sense.
  }
  \label{tab:conventions}
\end{table}

\begin{table}
  \centering
  \begin{tabular}{rlll}
    \textbf{Symbol} & \textbf{Value}       & \textbf{Units}           & \textbf{Description} \\
    $G$             & $6.67\times10^{-8}$  & cm$^3$ g$^{-1}$ s$^{-2}$ & Newton's gravitational constant \\
    $c$             & $3.00\times10^{10}$  & cm s$^{-1}$              & speed of light in vacuum \\
    $k_{\rm B}$     & $8.62\times10^{-11}$ & MeV K$^{-1}$             & Boltzmann's constant \\
    $G_{\rm F}$     & $2.3\times10^{-22}$  & cm MeV$^{-1}$            & Fermi coupling constant \\
    $G_{\rm F}/(\hbar c)^3$ & $1.17\times10^{-11}$ & MeV$^{-2}$       & ", naturalized units, $\hbar=c=1$ \\
    $\sin^2\theta_w$& 0.231                &                          & weak-mixing angle \\
    $m_e$           & 0.511                & MeV c$^{-2}$             & mass of electron \\
    $m_U$           & 939                  & MeV c$^{-2}$             & average nucleon mass, $(m_n+m_p)/2$ \\
    $M_\odot$       & $2.00\times10^{33}$  & g                        & solar mass \\
    $\hbar$         & $6.58\times10^{-22}$ & MeV s                    & reduced Planck's constant, $h/2\pi$ \\
  \end{tabular}
  \caption[Physical constants used in the text]{
    Some physical constants used in this text. Taken from the Particle Data
    Group, Particle Physics Booklet \citep{oliv2014-pdg}.
    I am only interested in scales, so I use three significant figures here.
    Note, the conversion factor between $G_{\rm F}$ in cgs and
    naturalized units is the naturalized length scale,
    $\hbar c/\varepsilon=1.97\times10^{-11}/(\varepsilon/{\rm MeV})$~cm.
  }
  \label{tab:constants}
\end{table}


\section{General Relativistic Hydrodynamics}
\label{sec:gr_hydro}
Here we introduce the equations describing the transport of particle number,
momentum, and energy by massive particles in the continuum limit, that is
\todo{specify ``continuum limit''}
a fluid.

Fluid dynamics are described in curved spacetime by the general relativistic
hydrodynamics equations:
\begin{align}
  \label{eqn:hydro_mass}
  \nabla_\alpha \rho u^\alpha &= 0 \\
  \label{eqn:hydro_mom}
  \nabla_\alpha T^{\alpha\beta} &= 0,
\end{align}
where $\nabla_\alpha$ is the covariant derivative,
$\rho$ the rest mass density,
$u^\alpha$ the four-velocity,
and $T^{\alpha\beta}$ the stress-energy tensor:
\begin{equation}
  \label{eqn:stress_energy}
  T^{\alpha\beta} = \rho h u^\alpha u^\beta + P \psi^{\alpha\beta},
\end{equation}
Here we have used $h=1+\epsilon+P/\rho c^2$ the specific enthalpy,
$\epsilon$ the specific energy,
$P$ the pressure, and
$\psi^{\alpha\beta}$ the inverse of the spacetime metric.
Eqns.~\ref{eqn:hydro_mass} and~\ref{eqn:hydro_mom} elegantly express the
conservation of mass and energy-momentum respectively.
\todo{add: ``in fact, familiar hydro in flat space in disguise...''}

Using the foliation of spacetime presented in Sec.~\ref{ssec:adm_metric}, we may
write these equations in conservative hyperbolic form,
using coordinate derivatives:
\begin{align}
  \label{eqn:adm_hydro_mass}
  \partial_t \rho_* + \partial_j \rho_* v^j &= 0 \\
  \label{eqn:adm_hydro_mom}
  \partial_t \tilde{S}_i + \partial_j(\alpha\sqrt{g}T^j_i) &=
  \frac{1}{2}\alpha\sqrt{g}T^{\mu\gamma}\partial_i\psi_{\mu\gamma} \\
  \label{eqn:adm_hydro_ener}
  \partial_t \tilde{\tau} + \partial_j(\alpha^2\sqrt{g}T^{0j}-\rho_*v^j) &=
  -\alpha\sqrt{g}T^{\mu\gamma}\nabla_\mu n_\gamma.
\end{align}
which express conservation of mass, momentum, and energy on the spatial slice.
In this form, we have introduced the conservative variables
\begin{align}
  \label{eqn:rhostar}
  \rho_*       &\equiv -\sqrt{g} n_\mu u^\mu \rho &&= \sqrt{g}W\rho \\
  \label{eqn:tildeS}
  \tilde{S_i}  &\equiv -\sqrt{g} n_\mu T^\mu_i    &&= \rho_* h u_i \\
  \label{eqn:tildetau}
  \tilde{\tau} &\equiv  \sqrt{g} n_\mu n_\gamma T^{\mu\gamma} - \rho_*
  &&= \rho_*(u^t h-1)-\sqrt{g}P,
\end{align}
where $W=\alpha u^t$ is the general relativistic Lorentz factor.

These are the hydrodynamic equations we solve to determine the evolution of
our model.
\todo{describe numerical solution}
\todo{add metric ev}

\section{General Relativistic Radiation Transport}
\label{sec:rad_transport}
Here we introduce the equations describing the transport of particle number,
momentum, and energy by massless particles in the continuum limit, that is,
\todo{specify ``continuum limit''}
by a field.

We begin with the familiar equations in flat spacetime, and then extend this
to a covariant formulation appropriate to situations with strong gravity.
We follow closely the treatment by \citet{miha1999-foundations}.

\subsection{Flat Spacetime}
The basic quantity of radiation is the specific intensity,
\begin{equation}
  \mathcal{I}(t,x^j;\nu,\Omega_k) \qquad \qquad
  {\rm erg}\,{\rm str}^{-1}\,{\rm cm}^{-2}\,{\rm Hz}^{-1}\,{\rm s}^{-1}, \nonumber
\end{equation}
where $\{t,x^j\}$ is the time and position of interest,
$\nu=c/\lambda$ the frequency\footnote{not the neutrino symbol as elsewhere},
related to the energy by Planck's constant $\varepsilon=h\nu$, and
$\Omega_k$ a normalized direction vector. Though a three-vector, $\Omega_k$
has only two degrees of freedom, which may be made explicit by parameterizing
in two angles; for example in cartesian components
$\Omega_k:=(\sin\alpha\cos\beta,\sin\alpha\sin\beta,\cos\alpha)$.

$\mathcal{I}$ is a time-dependent scalar field on a six-dimensional manifold,
$\{x^j,p_k\}$, with $p_k\equiv\frac{h\nu}{c}\Omega_k$.
We may visualize this space as an infinite bundle of rays---actually the
cotangent bundle $T^*M$ to the three-dimensional
spatial manifold $M^3:\{x^j\}$---and $\mathcal{I}$
assigns a weight to each ray, telling us how much of the radiation field has
momentum $p_k$ at point $x^j$. A ray is an element of $T^*M$,
and a beam is a subset of $T^*M$.

How much energy is in a beam?
The energy transported by a beam at $x^j$,
in a direction $\Omega_k$,
spread across a solid angle $\diff\Omega=\cos\alpha\,\diff\alpha\,\diff\beta$,
passing through a surface $\diff a$
oriented by normalized direction vector $a^i$,
within a frequency band $\diff \nu$ centered on $\nu$,
over a time $\diff t$ centered on $t$ is
\begin{align}
  \label{eqn:beam_energy_I_nu}
  \diff E &= \mathcal{I}(t,x^j;\nu,\Omega_k)\,
  \mu\, \diff\Omega\, \diff a\, \diff\nu\, \diff t \\
  \label{eqn:beam_energy_I_eps}
  &= \frac{1}{h} \mathcal{I}(t,x^j;p_k)\,
  \mu\, \diff\Omega\, \diff a\, \diff\varepsilon\, \diff t,
\end{align}
where $\mu\equiv a^i\Omega_i$ is the cosine of the angle between $a^i$ and
$\Omega^i$, and in Euclidean space $\Omega_i=\Omega^i$.
\todo{add schematic}

\subsubsection{The Particle Picture}
Much of what follows borrows from the particle picture of the radiation field.
By quantizing our cotangent bundle $T^*M$ into states (in other words,
quantum phase space), we may arrive at a particle formulation that is completely
analogous to the continuum picture.

The basic quantity of radiation in the particle picture is the distribution
function, $f(t,x^j;p_k)$, the number of particles per state in phase space.
Since each neutrino quantum state has a volume $h^{-3}$, where $h$ is Planck's
constant\footnote{the volume would be $2h^{-3}$ for photons, which exhibit two
distinct helicities},
the number of particles per phase space volume $\diff^3x\,\diff^3p$ is
\begin{equation}
  \label{eqn:dN_flatspace}
  \diff N = \frac{1}{h^3} f \,\diff^3 x\, \diff^3p.
\end{equation}
Because $h$ has units length$\times$momentum, we can see that $f$ is
dimensionless.

How much energy is in the radiation occupying a particular phase space volume?
The energy carried by $\diff N$ particles centered on $\{x^j,p_k\}$
is $\varepsilon\,\diff N$,
where $\varepsilon=pc$ and $p\equiv (p_ip^i)^{-1/2}$.
In spherical polar coordinates we may write
$\diff^3p=p^2\,\diff p\,\diff\Omega=\frac{\varepsilon^2}{c^3}\,\diff\varepsilon\,\diff\Omega$.
And we may identify $\diff^3x$
with the spatial volume occupied by a beam of particles
moving in direction $\Omega_k$,
passing through a surface $\diff a$
oriented by $a^i$,
over a time $\diff t$ centered on $t$,
so that $\diff^3x=c\,\diff t\,\mu\,\diff a$.
\todo{add schematic}
So in the particle picture, we have
\begin{equation}
  \label{eqn:beam_energy_f}
  \diff E = \frac{\varepsilon^3}{h^3c^2} f(t,x^j;p_k)\,
  \diff t\,\mu\,\diff a\, \diff\varepsilon\,\diff\Omega.
\end{equation}
Comparing Eqn.~\ref{eqn:beam_energy_I_eps} to Eqn.~\ref{eqn:beam_energy_f},
it's clear that
\begin{equation}
  \mathcal{I}=\frac{\varepsilon^3}{(hc)^2}f. \nonumber
\end{equation}
Below, we stick to the particle picture, formulating everything in terms of $f$.

\subsubsection{How $f$ Changes}
According to Liouville's Theorem,
absent any particle interactions with each other or with the medium, the density
of particles in phase space is constant along streamlines, $\Omega_k$. This may be
expressed as a continuity equation for the distribution function,
the advection equation:
\begin{align}
  \frac{1}{c}\partial_t f + \Omega^k\nabla_k f &= 0 \\
  \label{eqn:f_advection}
  p^\alpha\nabla_\alpha f &= 0,
\end{align}
where we have used the four-momentum in the second form.
\todo{confirm/rederive}
In flat space, we may write it $p^\alpha:=(\varepsilon/c,p^k)$.

When particles interact with the medium, we can describe the changes to $f$ with
source and sink terms.
Let $k(\varepsilon,f)$ be the opacity, or cross section per unit mass of the
medium, so that $k\rho f$ is the number of particles of energy
$\varepsilon$ absorbed or scattered per length traversed by the ray.
\todo{clarify 'ray' in particle picture}
And let $\eta(\varepsilon,f)$ be the emissivity, so that
$\eta/\varepsilon$ is the number of particles emitted per
unit volume at energy $\varepsilon$.
Opacity and emissivity are functions of $\varepsilon$ for obvious reasons, and
functions of $f$ because the processes of emission and absorption that these
objects describe may be stimulated by the radiation field.
Our equation of continuity, Eqn.~\ref{eqn:f_advection},
with source and sink terms is now
\begin{equation}
  \label{eqn:f_boltzmann}
  p^\alpha\nabla_\alpha f = \frac{1}{c}
  \left( \left(\frac{hc}{\varepsilon}\right)^2\eta - \varepsilon k \rho f\right).
\end{equation}
In cgs, $k$ is measured in cm$^2$~g$^{-1}$ and $\eta$ is measured in
erg~cm$^{-3}$. This is the Boltzmann equation of transport.

We may think of $k$ in terms of the cross-sections of absorption and scattering
processes; then we could write $\rho k = n(\sigma_{\rm abs}+\sigma_{\rm scatt})$,
where $n$ is the number density of the particles in the medium participating in
this process.
Also, note that we may analyze $k$ and $\eta$ by individual processes causing the
absorption, scattering, or emission; then these terms are sums.

We may find solutions by integrating this equation along rays using the length
parameterization $\diff s=c\,\diff t$, so that
$\Omega^i\nabla_i f=\diff f/\diff s$, with the rays defined by
the geodesic equations in Euclidean space:
\begin{equation}
  \label{eqn:geo_straight_ray}
  \frac{\diff x^j}{\diff s}=p^j, \qquad {\rm and} \qquad
  \frac{\diff p_k}{\diff s}=0.
\end{equation}
With this parameterization, Eqn.~\ref{eqn:f_boltzmann} becomes
\begin{equation}
  \label{eqn:f_boltzmann_ray}
  \frac{1}{c} \partial_t f + \frac{\diff}{\diff s} f =
  \frac{(hc)^2}{\varepsilon^3}\eta - \rho k f,
\end{equation}
\todo{derive rendering equation}

\subsubsection{Optical Depth}
A ray of radiation passing through a medium will attenuate due to scattering and
absorption. If there is no emission, and $f$ is changing quickly with respect to
fluid changes ($\lambda/c \ll T_{\rm dyn}$, in our context, where
$\lambda=(\rho k)^{-1}$ is the mean free path, and $T_{\rm dyn}$ is the timescale
of fluid changes presented in Sec.~\ref{sec:timescales})
Eqn.~\ref{eqn:f_boltzmann_ray} simplifies to
\begin{equation}
  \left(\frac{\diff}{\diff s} - \chi\right)f=0.
\end{equation}
where we have used $\chi\equiv\rho k$.
This is a linear first-order differential equation, which we may solve by
assuming an integrating factor $u(s)$ such that $u\chi=\diff u/\diff s$.
\todo{clarify}
We find that $u(s)=\exp\left(-\tau(A,B)\right)$, where
\begin{equation}
  \label{eqn:tau_flatspace}
  \Omega^j\nabla_j \tau = \chi, \qquad {\rm or} \qquad
  \tau \equiv \int_A^B \diff s\, \chi.
\end{equation}
While traveling from A to B, the number of particles in the beam attenuates by
the factor $\exp(-\tau)$.

\subsection{Curved Spacetime}
The preceding formalism may be extended to curved spacetimes by a suitable
covariant reformulation. Following \citet{lind1966-gr_boltzmann}\footnote{with
two slight modifications in that 1) we use the covariant four-momentum $p_\alpha$
to construct a more natural phase space volume element $\diff V\,\diff P$ below,
and 2) we include the projection of the momentum onto the observer's worldline,
$-p_\mu U^\mu$, in Eqn.~\ref{eqn:covariant_dV} where it seems to fit more
naturally},
and also drawing on \citet{ipse1968-gr_boltzmann},
we extend our manifold to four-dimensional spacetime
$M^4:\{x^\alpha\}$. The cotangent bundle, or phase space, is now
$T^*M:\{x^\alpha,p_\beta\}$.
We imagine an observer in this spacetime with four-velocity $U^\lambda$.

Our observer measures a spatial
volume $\diff V$ spanned by the three spacelike vectors
$\diff_1x^\alpha$, $\diff_2x^\beta$, $\diff_3x^\gamma$
(not differentials, but with names that look suggestively like differentials)
at $x^\mu$:
\begin{align}
  \label{eqn:covariant_dV}
  -U_\lambda\,\diff V &= \sqrt{-\psi} \, \epsilon_{\alpha\beta\gamma\lambda}\,
  \diff_1x^\alpha\, \diff_2x^\beta\, \diff_3x^\gamma (-p_\mu U^\mu) \\
  \label{eqn:dV}
  \diff V       &= \sqrt{-\psi} \, U^t \diff x\,\diff y\,\diff z\, (-p_\mu U^\mu),
\end{align}
where in the first equation, $\epsilon_{\alpha\beta\gamma\lambda}$ is the
totally antisymmetric tensor, and in the second equation we have
chosen to use the three cartesian basis vectors to span the volume.
$U^t \diff x\,\diff y\,\diff z$ is a scalar,
as are $\sqrt{-\psi}$ and the scalar product, which confirms that we have
constructed a Lorentz invariant spatial volume.

Our observer
measures a momentum volume $\diff P$ spanned by three timelike 1-forms
$\diff_1p_\alpha$, $\diff_2p_\beta$, $\diff_3p_\gamma$ at $p_\mu$
\begin{align}
  \label{eqn:covariant_dP}
  p^\lambda\, \diff P &= \frac{1}{\sqrt{-\psi}} \, \epsilon^{\alpha\beta\gamma\lambda}\,
  \diff_1p_\alpha\, \diff_2p_\beta\, \diff_3p_\gamma \\
  \label{eqn:dP}
  \diff P &= \frac{1}{\sqrt{-\psi}} \, \frac{1}{p^t} \diff p_x\, \diff p_y\, \diff p_z,
\end{align}
where again, in the second equation we have chosen to use the cartesian momentum
basis 1-forms.
Again $\diff p_x\, \diff p_y\, \diff p_z/p^t$ is a scalar,
which confirms that we have
constructed a Lorentz invariant momentum volume.

Then the invariant distribution function is the number of particle worldlines
passing through the invariant six-volume $\diff V\,\diff P$, scaled to a quantum
state by $h^3$:
\begin{align}
  \label{eqn:dN_covariant}
  \diff N &= \frac{1}{h^3} f\, \diff V \, \diff P\\
  \label{eqn:dN}
  &= \frac{1}{h^3} f\,
  \diff x\,\diff y\,\diff z\, \diff p_x\, \diff p_y\, \diff p_z\,
  \frac{U^t}{p^t} (-p_\mu U^\mu),
\end{align}
which is the covariant form of Eqn.~\ref{eqn:dN_flatspace}. The correspondence
may be seen by evaluating Eqn.~\ref{eqn:dN} in the locally flat Lorentz frame of
the observer, for whom $U^\alpha:=(1,0,0,0)$, and
$p_\mu U^\mu=p_tU^t=p_t=-p^t$.

\subsubsection{How $f$ Changes}
The covariant Boltzmann transport equation still takes the form of
Eqn.~\ref{eqn:f_boltzmann}, but $\nabla_\alpha$ now indicates covariant
differentiation (which, in the case of a scalar $f$, is simply
$\partial_\alpha$). But we may write the source and sink terms as
manifestly covariant objects:
\begin{equation}
  \label{eqn:f_boltzmann_covariant}
  p^\alpha \nabla_\alpha f = c \mathcal{E} - \frac{h}{c} \mathcal{A} f
\end{equation}
where the invariant spontaneous emission rate per rest mass of the medium is
$\mathcal{E}\equiv h^2\eta/\varepsilon^2$, measured in erg~s$^2$~cm$^{-3}$,
and the invariant opacity is
$\mathcal{A}\equiv \varepsilon k \rho/h$, measured in cm$^{-1}$~s$^{-1}$.

We may integrate this equation along rays using the length parameterization
$\diff\lambda=c\,\diff t/p^t$ with the rays defined by the geodesic equations in
curved spacetime:
\begin{equation}
  \label{eqn:geo_curved_ray}
  \frac{\diff x^\alpha}{\diff \lambda}=p^\alpha, \qquad {\rm and} \qquad
  \frac{\diff p_\beta}{\diff \lambda} =-\psi_{\alpha\beta}\Gamma^\alpha_{\mu\gamma} p^\mu p^\gamma,
\end{equation}
where $\Gamma^\alpha_{\mu\gamma}$ are the standard connection coefficients,
defined in Eqn.~\ref{eqn:christoffel}.

\subsubsection{Optical Depth}
The optical depth equation, Eqn.~\ref{eqn:tau_flatspace}, becomes
\begin{equation}
  \label{eqn:tau_covariant}
  p^\alpha\nabla_\alpha \tau = \frac{h}{c} \mathcal{A}, \qquad {\rm or} \qquad
  \tau \equiv \frac{h}{c} \int_A^B \diff \lambda\, \mathcal{A}.
\end{equation}
