\chapter{Introduction}
\label{chap:intro}

% Notation Conventions
% Brett Deaton -- Spring 2015

\section{Conventions}
\label{sec:conventions}

I adopt the standard general relativists metric signature convention
$(-,+,+,+)$.
Unless otherwise noted, I display equations in natural geometric units, so that
Newton's gravitational constant, $G$, and the speed of light, $c$, have unit
value.
\todo{not true, be consistent by chapter, and note it here}

I use abstract tensor notation in general: a tensor typeset with its indeces
represents the geometric object in all of its coordinate freedom. In other words
$T^{\alpha \beta}$ is identical to $T$, and is not $T$'s $\alpha\beta$-th
component.
In some equations, I use coordinate index notation accompanied with a warning.
Spatial tensors are given indices from the Latin alphabet ($a,b...$);
they exist on a 3-dimensional manifold. Spacetime tensors are given indices from
the Greek alphabet ($\alpha,\beta...$); they exist on a 4-dimensional manifold.
In instances of coordinate index notation, it follows naturally that
$\alpha,\beta...\in\{1,2,3\}$ and $a,b...\in\{0,1,2,3\}$.

Of course some symbols have labels other than their tensor indeces.
In particular, I use subscript labels before tensor indeces (e.g.\ the
momentum of a neutrino is $p_{\nu \alpha}$). I allow context to clarify when
an index ranges over some set other than dimension (e.g.\ the momentum of the
i-th flavor of neutrino is $p_{\nu_i \alpha}$). The subscript $\nu$ is always
used as a label, not as a tensor index.

\begin{table}
  \centering
  \begin{tabular}{rll}
    \textbf{Symbol}       & \textbf{Object}           & \\%\textbf{Additional Notes} \\
    $n_b$                 & baryon number density     & \\
    $\rho$                & rest density              & $\rho \equiv m_U n_b $ \\
    $\epsilon$            & specific internal energy  & not including the mass energy \\
    $h$                   & specific enthalpy         & $h=1+\epsilon+P/\rho$\\
    $P$                   & pressure                  & \\
    $T$                   & temperature               & \\
    $T^{\alpha \beta}$    & stress-energy tensor      & \\
    $Y_e$                 & electron fraction         & $Y_e=n_e/n_b$\\
    $\rho_*$              & density evolution variable& $\rho_*=\sqrt{g}W\rho$ \\
    $\tilde\tau$          & energy evolution variable & $\tilde\tau=\rho_*(hW-1)-\sqrt{g}P$ \\
    $\tilde S$            & momentum evolution variable & $\tilde S=\rho_*hu_i$\\
    $v^i$                 & Lagrangian velocity       & a.k.a.\ transport velocity, $v^i=u^i/u^t$ \\
    $u^\alpha$            & fluid four-velocity       & $u^a=dx^a/d\tau$ \\
    $U^\alpha$            & observer four-velocity    & used if we want to distinguish from $u^\alpha$ \\
    $d\tau$               & spacetime interval        & $d\tau=-dx^a dx^b \psi_{ab}$ \\
    $W$                   & Lorentz factor            & $\alpha u^t$ (if no gravity $W=1/\sqrt{1-v^2}$) \\
    $\tau$                & optical depth             & \\
    $Q_\nu$               & local energy emission rate& \\
    $R_\nu$               & local lepton number rate  & \\
    $\mu$                 & chemical potential        & $\mu=T\eta$ \\
    $p_\alpha$            & momentum 1-form           & \\
    $\varepsilon$         & neutrino energy           & used for asymptotic and local \\
    $\psi_{\alpha\beta}$  & spacetime metric          & \\
    $g_{ij}$              & spatial metric            & \\
    $\beta^i$             & shift                     & where $\beta_i = \beta^j g_{ij}$ \\
    $\alpha$              & lapse                     & \\
  \end{tabular}
  \caption[Symbols used in the text]{
    Some symbols used in this text.
    Indices are used in the ``Symbol'' column in an abstract sense, and in the
    ``Additional Notes'' column in a component sense.
  }
  \label{tab:conventions}
\end{table}

\begin{table}
  \centering
  \begin{tabular}{rlll}
    \textbf{Symbol} & \textbf{Value}       & \textbf{Units}           & \textbf{Description} \\
    $G$             & $6.67\times10^{-8}$  & cm$^3$ g$^{-1}$ s$^{-2}$ & Newton's gravitational constant \\
    $c$             & $3.00\times10^{10}$  & cm s$^{-1}$              & speed of light in vacuum \\
    $k_{\rm B}$     & $8.62\times10^{-11}$ & MeV K$^{-1}$             & Boltzmann's constant \\
    $G_{\rm F}$     & $2.3\times10^{-22}$  & cm MeV$^{-1}$            & Fermi coupling constant \\
    $G_{\rm F}/(\hbar c)^3$ & $1.17\times10^{-11}$ & MeV$^{-2}$       & ", naturalized units, $\hbar=c=1$ \\
    $\sin^2\theta_w$& 0.231                &                          & weak-mixing angle \\
    $m_e$           & 0.511                & MeV c$^{-2}$             & mass of electron \\
    $m_U$           & 939                  & MeV c$^{-2}$             & average nucleon mass, $(m_n+m_p)/2$ \\
    $M_\odot$       & $2.00\times10^{33}$  & g                        & solar mass \\
    $\hbar$         & $6.58\times10^{-22}$ & MeV s                    & reduced Planck's constant, $h/2\pi$ \\
  \end{tabular}
  \caption[Physical constants used in the text]{
    Some physical constants used in this text. Taken from the Particle Data
    Group, Particle Physics Booklet \citep{oliv2014-pdg}.
    I am only interested in scales, so I use three significant figures here.
    Note, the conversion factor between $G_{\rm F}$ in cgs and
    naturalized units is the naturalized length scale,
    $\hbar c/\varepsilon=1.97\times10^{-11}/(\varepsilon/{\rm MeV})$~cm.
  }
  \label{tab:constants}
\end{table}


\section{General Relativistic Hydrodynamics}
\label{sec:gr_hydro}
Here we introduce the equations describing the transport of particle number,
momentum, and energy by massive particles in the continuum limit, that is
\todo{specify ``continuum limit''}
a fluid.

Fluid dynamics are described in curved spacetime by the general relativistic
hydrodynamics equations:
\begin{align}
  \label{eqn:hydro_mass}
  \nabla_\alpha \rho u^\alpha &= 0 \\
  \label{eqn:hydro_mom}
  \nabla_\alpha T^{\alpha\beta} &= 0,
\end{align}
where $\nabla_\alpha$ is the covariant derivative,
$\rho$ the rest mass density,
$u^\alpha$ the four-velocity,
and $T^{\alpha\beta}$ the stress-energy tensor:
\begin{equation}
  \label{eqn:stress_energy}
  T^{\alpha\beta} = \rho h u^\alpha u^\beta + P \psi^{\alpha\beta},
\end{equation}
Here we have used $h=1+\epsilon+P/\rho c^2$ the specific enthalpy,
$\epsilon$ the specific energy,
$P$ the pressure, and
$\psi^{\alpha\beta}$ the inverse of the spacetime metric.
Eqns.~\ref{eqn:hydro_mass} and~\ref{eqn:hydro_mom} elegantly express the
conservation of mass and energy-momentum respectively.
\todo{add: ``in fact, familiar hydro in flat space in disguise...''}

Using the foliation of spacetime presented in Sec.~\ref{ssec:adm_metric}, we may
write these equations in conservative hyperbolic form,
using coordinate derivatives:
\begin{align}
  \label{eqn:adm_hydro_mass}
  \partial_t \rho_* + \partial_j \rho_* v^j &= 0 \\
  \label{eqn:adm_hydro_mom}
  \partial_t \tilde{S}_i + \partial_j(\alpha\sqrt{g}T^j_i) &=
  \frac{1}{2}\alpha\sqrt{g}T^{\mu\gamma}\partial_i\psi_{\mu\gamma} \\
  \label{eqn:adm_hydro_ener}
  \partial_t \tilde{\tau} + \partial_j(\alpha^2\sqrt{g}T^{0j}-\rho_*v^j) &=
  -\alpha\sqrt{g}T^{\mu\gamma}\nabla_\mu n_\gamma.
\end{align}
which express conservation of mass, momentum, and energy on the spatial slice.
In this form, we have introduced the conservative variables
\begin{align}
  \label{eqn:rhostar}
  \rho_*       &\equiv -\sqrt{g} n_\mu u^\mu \rho &&= \sqrt{g}W\rho \\
  \label{eqn:tildeS}
  \tilde{S_i}  &\equiv -\sqrt{g} n_\mu T^\mu_i    &&= \rho_* h u_i \\
  \label{eqn:tildetau}
  \tilde{\tau} &\equiv  \sqrt{g} n_\mu n_\gamma T^{\mu\gamma} - \rho_*
  &&= \rho_*(u^t h-1)-\sqrt{g}P,
\end{align}
where $W=\alpha u^t$ is the general relativistic Lorentz factor.

These are the hydrodynamic equations we solve to determine the evolution of
our model.
\todo{describe numerical solution}
\todo{add metric ev}

\section{General Relativistic Radiation Transport}
\label{sec:rad_transport}
Here we introduce the equations describing the transport of particle number,
momentum, and energy by massless particles in the continuum limit, that is,
\todo{specify ``continuum limit''}
by a field.

We begin with the familiar equations in flat spacetime, and then extend this
to a covariant formulation appropriate to situations with strong gravity.

\subsection{Flat Spacetime}
\todo{cite Mihalas-Mihalas}
The basic unit of radiation is the specific intensity,
\begin{equation}
  \mathcal{I}(t,x^j;\nu,\Omega_k) \qquad \qquad
  {\rm erg}\,{\rm str}^{-1}\,{\rm cm}^{-2}\,{\rm Hz}^{-1}\,{\rm s}^{-1}, \nonumber
\end{equation}
where $\{t,x^j\}$ is the time and spatial position of interest,
$\nu=c/\lambda$ the frequency\footnote{not the neutrino symbol as elsewhere},
related to the energy by Planck's constant $\varepsilon=h\nu$, and
$\Omega_k$ a normalized direction vector. Though a three-vector, $\Omega_k$
has only two degrees of freedom, which may be made explicit by parameterizing
in two angles; for example in cartesian components
$\Omega_k:=(\sin\alpha\cos\beta,\sin\alpha\sin\beta,\cos\alpha)$.

$\mathcal{I}$ is a time-dependent scalar field on a six-dimensional manifold,
$\{x^j,p_k\}$, with $p_k\equiv\frac{h\nu}{c}\Omega_k$.
We may visualize this space as an infinite bundle of rays---actually the
cotangent bundle $T^*M$ to the spatial manifold $M:\{x^j\}$---and $\mathcal{I}$
assigns a weight to each ray, telling us how much of the radiation field has
momentum $p_k$ at point $x^j$. A ray is an element of $T^*M$,
and a beam is a bundle of rays, or a subset of $T^*M$.

How much energy is in a beam?
The energy transported by beam at $x^j$,
in a direction $\Omega_k$,
spread across a solid angle $\diff\Omega=\cos\alpha\,\diff\alpha\,\diff\beta$,
passing through a surface $\diff a$
oriented by normalized direction vector $a^i$,
within a frequency band $\diff \nu$,
over a time $\diff t$ centered on $t$ is
\begin{align}
  \label{eqn:beam_energy_I_nu}
  \diff E &= \mathcal{I}(t,x^j;\nu,\Omega_k)\,
  \mu\, \diff\Omega\, \diff a\, \diff\nu\, \diff t \\
  \label{eqn:beam_energy_I_eps}
  &= \frac{1}{h} \mathcal{I}(t,x^j;\varepsilon,\Omega_k)\,
  \mu\, \diff\Omega\, \diff a\, \diff\varepsilon\, \diff t,
\end{align}
where $\mu\equiv a^i\Omega_i$ is the cosine of the angle between $a^i$ and
$\Omega^i$, and in Euclidean space $\Omega_i=\Omega^i$.



\subsection{Curved Spacetime}
