\chapter{Properties of Neutrino Emission}
\label{chap:ray_tracing}

The leakage approximation used in Chap.~\ref{chap:leakage} gives us a beginning
understanding of the cooling effects of neutrinos on the accretion disk, or
how the disk changes as it radiates. Through that approximation we have been able
to estimate changes in the energy and composition of the fluid at the position
$x^\alpha$ due to the emission of neutrinos at $x^\alpha$.

However, the leakage approximation teaches us very little about the properties of
that emission. We \emph{can} extract some rough global measures:
1) the sum of the energy losses from all positions in the disk (i.e.\
the volume integral of $Q_{\nu_i}$ defined in Sec.~\ref{sec:leakage})
gives us the total energy luminosity, $L_{\nu_i}$, at any epoch,
2) likewise the sum of $R_{\nu_i}$ gives us the total lepton number luminosity
(also denoted $R_{\nu_i}$),
3) and the ratio of these two luminosities gives us the average neutrino energy,
$\langle E_{\nu_i} \rangle$. These three calculations are presented in
Fig.~\ref{fig:neutrinos_by_species}.

But to answer some of the questions presented in Chap.~\ref{chap:overview}, we
need to know more about the radiation field outside of the disk, particularly
how it varies with position, $x^\alpha$, and how it is distributed in energy and
angle, $p_k$.
That is, we need to develop an estimate of the neutrino distribution function,
$f_{\nu_i}(x^\alpha;p_k)$.

Where the mean free path of the neutrinos is long with respect to local fluid
\todo{show this is the case}
scales at $x^\alpha$, $f(x^\alpha)$ is dependent upon the state of matter far
away from $x^\alpha$. In lieue of solving the Boltzmann equation in full
\todo{ref eqn in Chap.~\ref{chap:intro}}
(which would involve an evolution of a scalar field like those presented in
Chap.~\ref{chap:leakage} for fluid and metric fields, but over a 6-dimensional
phase space manifold with coordinates $\{x^j,p_k\}$, and therefore
computationally out of reach today),
we can capture this nonlocality with a ray tracing solution.
In ray tracing, $f(x^\alpha;p_k)$ is computed by tracing a neutrino trajectory
backwards from $x^\alpha$ with a momentum $p_k$, keeping track of additions
and depletions to the neutrino number density along that trajectory.
Each ray yields an approximate solution to the Boltzmann equation at a
single point on the phase space manifold of interest. A ray
tracing solution to the Boltzmann equation places an observer at $x^\alpha$,
points him in the direction $p_k$, and asks ``how many neutrinos from that
direction?''
This approximation neglects additions to the neutrino number density due to
scattering into the line of sight (the kind of scattering that gives to the
moon or a streetlamp a halo on a foggy night).
\todo{ref later discussion}

In this chapter I present a ray-tracing algorithm for estimating
$f_{\nu_i}(x^\alpha;p_k)$ in relativistic numerical spacetimes and fluid
distributions, I calculate $f_{\nu_i}$ describing neutrino
emission from the model accretion disk from Chap.~\ref{chap:leakage},
and I use these $f$s to estimate $q_{\nu\bar{\nu}}$, the heating due to
neutrino-antineutrino annihilation in the funnel of the disk.
Because of large uncertainties,
this last calculation should be considered a proof-of-concept for future
calculations. It may also be considered a back-of-the-envelope answer to
the question, ``Can a neutron star--black hole merger drive a gamma ray burst
jet by neutrino processes alone?''

Sec.~\ref{sec:f_algorithm} gives a technical overview of the ray tracing
algorithm and two code tests. Sec.~\ref{sec:f_this_case} presents the distribution
function of the three 'leakage' neutrino species around this model accretion
disk. Secs.~\ref{sec:q_algorithm}~and~\ref{sec:q_this_case} describe the
algorithm for and results from a neutrino-antineutrino annihilation heating
calculation for this model.

\section{Calculating $f_\nu$: Ray Tracing Algorithm}
\label{sec:f_algorithm}

In the most comprehensive treatment, a ray-tracing solution for $f$ is calculated
by solving the rendering equation backwards along a trajectory terminating at
$x^a$ with momentum $p_k$:
\todo{derive rendering eqn, or check/ref below}
\begin{equation}
  \label{eqn:rendering}
  f(x^\alpha;p_k) = \int_{x^a}^{\rm far\,away} \diff \lambda \, \mathcal{E}(p_k)
  \exp\left(-\int_{x^a}^\lambda  \diff \lambda' \, \mathcal{A}(p_k)\right)
\end{equation}
where $\mathcal{E}$ and $\mathcal{A}$ are the invariant emissivity and opacity
defined in Chap.~\ref{chap:intro}.
\todo{ref eqns}
For isotropic scattering, $p_k$ only enters $\mathcal{E}$ and $\mathcal{A}$ in the
form $-p_\beta u^\beta$, the neutrino energy in the rest frame of the fluid.
Furthermore, $p_\beta$ may be calculated from $p_i$ by invoking the neutrino mass,
$-m_{\nu_i}^2=p_\beta p^\beta$.
All the neutrino masses are vanishingly small in the case of nuclear
accretion disks, with fluid energy scales of a few MeV. So in the following we
set $m_{\nu_i}=0$.

As $f$ accumulates along the ray, the integrand in Eqn.~\ref{eqn:rendering}
gets attenuated by the exponential term, which is the optical depth defined in
Chap.~\ref{chap:intro}.
\todo{ref eqn}
At large optical depths, the integrand vanishes exponentially, because fluid
properties on the far side of opaque matter cannot affect the neutrino
distribution function on this side. The exponential behavior of the integrand
leads us to pose a further approximation, that all neutrinos are emitted from
a single point along the trajectory, the neutrinosurface.

In this treatment, the ray is traced backwards until the optical depth becomes
large ($\tau\sim1$). This defines the neutrinosurface at $x_0^\alpha$. The
neutrinos are assumed to decouple from the matter at this surface, and so f
is simply the Fermi-Dirac distribution:
\begin{equation}
  \label{eqn:f_fermi_dirac}
  f(x^\alpha;p_k) =
  \left(\exp\left(\frac{p_\beta u^\beta}{T(x_0^\alpha)}-\eta \right)+1\right)^{-1}
\end{equation}
where $T(x_0^\alpha)$ is the fluid temperature at the neutrinosurface, and $\eta$
is the neutrino chemical potential scaled by temperature. We use $\eta$ estimated
by the leakage scheme (Eqn.~\ref{mu_nu}).
\todo{other choices for $\eta$}

\section{$f_\nu$ for this Case}
\label{sec:f_this_case}

\section{Caclculating $q_{\nu \bar{\nu}}$: Integration Algorithm}
\label{sec:q_algorithm}

\section{$q_{\nu \bar{\nu}}$ for this case}
\label{sec:q_this_case}
