\documentclass[12pt]{report}
% Intended to be built with pdflatex.
% Labeling convention example: \label{ssec:acknowledge}.
%   Specifically, use (<=4)-letter class (e.g. chap, sec, ssec, fig, etc.),
%   followed by a colon, followed by a shortish underscore separated name
%   (e.g. intro, nu_energy_spectrum).

% packages
\usepackage{amssymb}              % for \gtrsim
\usepackage{amsmath}              % for \text, align, split ...
\usepackage{booktabs}             % for \toprule
\usepackage{graphicx}             % for \includegraphics
\usepackage{hyperref}             % for \url and \href
\usepackage{mbdeaton}             % include my special macros
\usepackage{natbib}               % allows me to define the citation style
\usepackage{setspace}             % for \doublespacing
\usepackage{subcaption}           % for subfigure
\usepackage{wsudissertation}      % definitions for WSU-specific formatting

% initializations
\doublespacing                    % use \begin{singlespace} ... \end{singlespace}
                                  % for stretches of monospace text
\bibliographystyle{abbrvnat}                 % set bibliography style
\setcitestyle{authoryear,open={(},close={)}} % set citation style not numeric

% *******************************************************************************
\begin{document}
\pagenumbering{roman}

% *******************************************************************************
\begin{titlepage}
  \begin{singlespace}
    \begin{center}
      {\uppercase{
          Neutrinos in Mergers of\\
          \bigskip
          Neutron Stars with\\
          \bigskip
          Black Holes}}\\
      \vspace{1.31in}
      By\\
      \bigskip
      \uppercase{Michael Brett Deaton}\\
      \vspace{2.65in}
      A dissertation submitted in partial fulfillment of\\
      the requirements for the degree of\\
      \bigskip
      \uppercase{Doctor of Philosophy}\\
      \bigskip \bigskip \bigskip
      \uppercase{Washington State University}\\
      Department of Physics and Astronomy\\
      \bigskip
      \uppercase{August 2015}\\
      \bigskip \bigskip
      $\copyright$ Copyright by MICHAEL BRETT DEATON, 2015\\
      All Rights Reserved
    \end{center}
  \end{singlespace}
\end{titlepage}
\newpage

% *******************************************************************************
\thispagestyle{empty} %\addtocounter{page}{-1}
\begin{center}
  \begin{singlespace}
    \null % necessary to get \vfill to work without initial text
    \vfill
    $\copyright$ Copyright by \uppercase{Michael Brett Deaton}, 2015\\
    All Rights Reserved
  \end{singlespace}
\end{center}
\newpage

% *******************************************************************************
\begin{singlespace}
  \noindent
  \vspace{1.5in}

  \noindent To the Faculty of Washington State University:\\
  
  The members of the Committee appointed to examine the dissertation of
  \medskip
  \uppercase{Michael Brett Deaton}
  find it satisfactory and recommend that it be accepted.
  
  \begin{flushright}
    \signhere{Matthew D.\ Duez, Ph.D., Chair}\\
    \signhere{Sukanta Bose, Ph.D.}\\
    \signhere{Guy Worthey, Ph.D.}
  \end{flushright}
\end{singlespace}
\newpage

% *******************************************************************************
\begin{center}
  \uppercase{Acknowledgements}
  \addcontentsline{toc}{section}{Acknowledgements} % insert into TOC
\end{center}

  \bigskip
  During his first year as a professor (coinciding with his first
  year as a dad), my advisor, Matt Duez, coded up the leakage approximation
  at the center of Chap.~\ref{chap:leakage} of this thesis. I am grateful
  for this and for his mentorship in research and teaching. His reasoning is
  always clear and enlivening.
  In addition, Matt and Francois Foucart wrote the core of the general
  relativistic hydrodynamics code that I used. Without the robust simulation
  tool they developed, I could not have done this work.
  Andy Bohn wrote the ray-tracing code our collaboration uses to find event
  horizons. His clear coding style and generous correspondence made it easy
  for me to co-opt large parts of his work for my own ray-tracing code.

  Matt, Francois, and Evan O'Connor passed
  on their knowledge of hydrodynamics, thermodynamics, and radiation transport
  through discussions, emails, video conferences, and during my visit to the
  University of Toronto.
  Many other colleagues gave me a bit of their time and knowledge: particularly
  Jeff Kaplan, Mark Scheel, B\'ela Szil\'agyi, Curran Muhlberger,
  Daniel Hemberger, and Saul Teukolsky.

  Together Matt and I co-wrote \citealt{deat2013-leakage}
  (Chap.~\ref{chap:leakage} of this thesis).
  Francois, Evan, and Christian Ott
  thoroughly read and commented on each draft,
  and my writing grew in stature and wisdom and in favor with God and man.

  My colleague and friend Fatemeh Hossein-Nouri helped me in countless debugging
  sessions, shared her relativistic intuition, and introduced me to Hafiz.
  My friends Dan Helms, Ryan Niemeyer, Dusty Vaughn, Phillip Jacobs, Riley
  Rex, Josef Felver, and Joseph Halbert were happy and strong.
  My mom and dad, Mike and JoEtta, were early and late guides.
  One of my favorite undergraduate professors, Mike Sadler, crossed paths with me
  at every April APS Meeting. Once, after one of my talks,
  he told me directly and gently, ``I didn't understand how your findings could
  be tested,'' and I've taken that to heart.

  Finally, my wife and companion, Darci, has lit up the last half of my graduate
  studies with her own probing intellect and love. This work bears the imprint
  of some of her questions.

\newpage

% *******************************************************************************
\begin{center}
  \begin{singlespace}
    \addcontentsline{toc}{section}{Abstract} % insert into TOC
    \label{ssec:abstract}

    {\uppercase{
        Neutrinos in Mergers of\\
        \bigskip
        Neutron Stars with\\
        \bigskip
        Black Holes}}\\
    \bigskip
    Abstract\\
    \bigskip \bigskip \bigskip
    by Michael Brett Deaton, Ph.D.\\
    Washington State University\\
    August 2015\\
    \bigskip \bigskip \bigskip
    Chair: Matthew D.\ Duez
  \end{singlespace}
\end{center}
  
Mergers of a neutron star and a black hole are interesting because of the dual
complexity of the black hole's strong gravity and the neutron star's
nuclear-density fluid.
Mergers can yield short-lived nuclear accretion disks, emitting copious neutrinos.
This radiation
may change the thermodynamic state of the disk itself,
may drive an ultrarelativistic jet of electrons and positrons,
may oscillate in its flavor content,
may irradiate surrounding matter, playing a role in nucleosynthesis,
and may be detected directly.

In this thesis I present a model of such a merger, its remnant accretion disk,
and its neutrino emission.
In particular, we evolve a neutron star--black hole merger through $\sym100$~ms,
solving the full general relativistic hydrodynamics equations,
from inspiral through merger and accretion epochs.
We treat the neutrinos approximately, using a leakage framework,
which accounts for local energy losses and composition drift in the fluid due
to escaping neutrinos.
We use geodesic ray tracing on a late time slice of the model to calculate the full
spatial-, angular-, and energy-dependence of the neutrino distribution function
around the accretion disk.
This distribution then serves in a computation of the energy available to
form a jet via neutrino-antineutrino annihilation in the disk funnel.
In this scenario, we find that enough energy is deposited
to drive a jet of short-gamma-ray-burst-energy by neutrino processes alone.

\newpage

\tableofcontents
\newpage

\listoftables
\addcontentsline{toc}{section}{List of Tables} % insert into TOC
\newpage

\listoffigures
\addcontentsline{toc}{section}{List of Figures} % insert into TOC
\newpage

\begin{center}
  \null
  \vspace{2.7in}
  \bigskip
  To MLD \& JHD,\\
  KDG \& JFG,\\
  and DLD.\\
  For many many things,\\
  but most of all\\
  for love.
  \newpage
\end{center}
\pagenumbering{arabic}

% Overview
\chapter{Overview}
\label{chap:overview}

\section{Why this Study?}
\label{sec:why_this_study}

Neutron stars and stellar-mass black holes form out of massive stars when they
exhaust their nuclear fuel. White dwarfs form in the same way, but from less
massive stars (with $M\lesssim 8\,M_{\odot}$), like our sun.
For stars of $M\gtrsim 8\,M_{\odot}$, generally the more massive
($M\gtrsim 20\,M_{\odot}$ end up as black holes, and the less massive
($M\lesssim 20\,M_{\odot}$) end up as neutron stars.
\citep{woos2002-review_stellar_evolution}

Unlike main-sequence stars, which are supported against gravitational collapse
by thermal pressure due to nuclear burning, neutron stars and white dwarfs
don't burn fuel; their composition is fixed.
Their resistance to collapse, in the language of quantum mechanics, is due
to degeneracy pressure, a physical manifestation of the uncertainty principle.
Neutron stars are supported by degenerate neutrons, while white dwarfs are
supported by degenerate electrons. In contrast to both of these, black holes
are the unique objects which, in equilibrium, are not supported against
collapse. They are so compact gravity overwhelms all of Nature's repulsive
forces, and their surfaces fall inside a causal horizon. After this, gravity
enforces a one-way flow of matter and radiation inward.

Both neutron stars and white dwarfs are limited by a maximum mass above which
gravitational pressure overwhelms degeneracy pressure. Neutron stars with masses
greater than $\sym2\,M_\odot$, or white dwarfs with masses greater than
$\sym1.4\,M_\odot$, cannot exist in equilibrium: they are unstable against
collapse to a black hole.

Neutron stars and black holes are called compact objects because their enclosed
mass per surface radius, or compactness $\mathcal{C}\equiv G M/c^2 R$, is very
large:
$\mathcal{C}_{\rm BH}=0.5$ and
$\mathcal{C}_{\rm NS}\sim0.15$,
whereas $\mathcal{C}_{\rm WD}\sim10^{-4}$, and $\mathcal{C}_{\odot}\sim10^{-6}$.
\todo{check $\mathcal{C}_{\rm WD}$ and $\mathcal{C}_{\odot}$}
Because gravity plays a greater role at lesser distances from a massive object,
a body's compactness is a measure of the significance of gravity
in explaining changes to that body and its immediate environment.
For neutron stars and black holes gravity is very important.

Binaries of compact objects may form by two evolutionary channels: from the
double supernovae of a pair of main sequence stars already in binary orbit,
or from dynamical capture in dense stellar regions, like globular clusters,
where near-flybys are common.
\todo{cite ... maybe Stephens+}
In either case, when two compact objects orbit each other closely, gravitational
theory predicts \citep{eins1916-integrate_field_eqns,eins1918-grav_waves}
and observations confirm \citep{will2014-review}
that their orbit slowly shrinks. Orbital energy and angular momentum
are transported away from the system by gravitational waves.

Compact binaries come in three combinations: black hole--black hole \bhbh,
neutron star--black hole \nsbh, and neutron star--neutron star \nsns.
At the writing of this thesis,
the only known compact binary systems are ten neutron star--neutron star
binaries \nsns \citep{post2014-evolution_compact_binaries}; no compact binaries
involving a black hole have yet been discovered.
Becuase all the known systems comprise neutron stars---which are radio-visible as
pulsars, blinking at exquisitely regular intervals---excellent orbital timing
information is available for them.

In fact, we can watch the gravitational drain of orbital energy in many
cases, the most famous being the Hulse-Taylor neutron star binary
\citep{huls1975-discovery}.
Since radio astronomers began tracking it in 1974, its 7.75~hr orbit has
shortened by a few milliseconds \citep{weis2010-hulse_taylor_timing}.
% Estimate based on orbital params on 52984.0 MJD (12.11.03):
% \Delta P = P_b*\dot{P_b}*24*3600 [= 70 ns per orbit]
% N        = 365.24/P_b            [= 1130 orbits per yr at end 20th Cent]
% \Delta T = N*40*\Delta P         [= -3 ms per 40 yrs at end 20th Cent]
At this rate, the two neutron stars will touch in $\sym$300~million years,
% from simple integration:
% T_merge  = P_b/\Delta T          [= 300 Myrs]
at which time the quiescent binary will go through a number of violent changes.
This is called merger.

\subsection{How Common are Neutron Star--Black Hole Mergers?}
\label{ssec:how_common}

Astrophysicists use the small collection of known neutron star--neutron star
binaries \nsns to predict the merger rate of such systems in the local universe.
From the ? known systems at the time of publication, Kalogera et al.\
\citeyearpar{kalo2004-bns_merger_rate, kalo2004-erratum} extrapolated ?
\todo{fill in}
mergers per Milky-Way-equivalent-galaxy per million years.
Neither black hole--black hole \bhbh nor neutron star--black hole \nsbh
merger rates can be extrapolated from observations.

However, at least one possible progenitor to such a system is known: Cygnus X-1.
Cygnus X-1 (discovered by ?
\todo{citealt Bowyer+ 1965}
)
is an x-ray source whose energetic and rapidly-flickering emission
is well-modeled as a $\sym15\,M_\odot$ black hole orbiting a $\sym20\,M_\odot$
main-sequence companion.
\todo{cite Wong+ 2012}
Several evolution scenarios are possible, with a small probability of forming
a compact binary in close enough orbit to merge within the age of the Universe
(about 1\% likelihood according to ?).
Dominik et al.\
\todo{citet Belczynski+ 2011}
explore these scenarios, and from their relative likelihoods, draw conclusions
about the rate of neutron star--black hole \nsbh mergers in our galaxy.
In ? million years,
\todo{check}
the companion star will exhaust its nuclear fuel and its core will collapse,
triggering a supernova. It may form a neutron star, which, if the asymmetry of
the explosion isn't so great as to gravitationally unbind it from the black hole,
will yield a neutron star--black hole \nsbh binary. In less than 1\% of the possible
scenarios, the compact binary forms close enough to merge within the age of the
Universe. If the only formation channel is through
binaries like the one that formed the Cygnus X-1 system, there must be very few
neutron star--black hole \nsbh binaries in our galaxy, and the merger rate observable
by Advanced LIGO is 1 detection every 30-250~years.
\todo{citet Belczynski+ 2011}
\todo{introduce LIGO}

%\subsubsection{Theoretical Answers}
%\label{sssc:theory}
The same astrophysicists have pursued a more comprehensive approach to this
calculation through population synthesis studies: the modeling of large numbers
of stars representative of populations observed in our own and nearby galaxies.
The models attempt to capture the particular nuclear effects, gravitational
interactions, stellar evolutions, and supernova dynamics of millions of binaries,
to reach an ultimate conclusion about the compact remnants in each system:
binary or no? and if yes, is the binary orbiting closely enough to merge in
$10^{10}$~yrs, the age of our Universe?
Population synthesis models are attended by large uncertainties,
but for Milky-Way-equivalent galaxies, at the current of the Universe,
Dominik et al. 2013\
\todo{fix cite}
predict ? neutron star--black hole \nsbh mergers per galaxy per Myr.
\todo{compared to \nsns, much more common in low-z galaxies}
This rate gives us a theoretical answer to the question, ``how common are
neutron star--black hole mergers?''.

%\subsubsection{Observational Answers}
%\label{sssc:observation}
These theoretical predictions may be complemented or challenged by observations.
In the next few years, astronomers expect to record the final seconds to minutes
\todo{check}
of the gravitational wave signal emitted by a compact binary
as it merges, using gravitational wave interferometers like LIGO, Virgo, and Geo.
\todo{cite AdvLIGO white paper}
Or if matter is at hand, that is, if one or both of the objects is a neutron
star, they may observe electromagnetic or neutrino signals. Let's discuss
electromagnetic signals here; in Sec.~\ref{ssec:neutrino_roles}, we will
turn to neutrinos.

A diverse family of electromagnetic signals is possible from a neutron
star--neutron star \nsns or neutron star--black hole merger \nsbh:
\begin{itemize}
  \item Kilonova: Some of the neutron star material may be ejected and escape
    from the system. As this neutron-rich matter expands and cools, exotic heavy
    nuclei form like ice crystals in a winter pond. These nuclei only form in a
    low-energy bath of free neutrons via rapid neutron capture, the r-process.
    Later, these unstable nuclei decay and, like a nuclear power stack,
    \todo{stack?}
    heat the ejecta, causing it to emit photons with a
    thermal spectrum, at optical or infrared frequencies.
    \citep{
      robe2011-transients,
      metz2012-most_promising,
      kase2013-opacities}.
    \todo{more cite}

    Recently space-based infrared telescopes have observed short glows that
    fit these theoretical predictions, and are promising candidates of this
    theoretically-predicted event.
    \todo{cite Berger, and the other group ~2014}
  \item Late radio afterglow: The distant neighborhood of the merger, the
    circumburst medium, is often denser than the binary's immediate environment.
    \todo{often?}
    The circumburst medium may be simply the interstellar medium.
    \todo{what else?}
    As the ejecta rams into the denser medium, it experiences hydrodynamic
    shocks, amplifying any seed magnetic fields, and causing the plasma to
    emit radio-frequency synchrotron radiation.
    \todo{synchrotron? how long? cite}
    \todo{cite observations}
  \item Early x-ray afterglow: If the progenitor is a neutron star--neutron
    star binary \nsns, the immediate product of the merger may be
    a supermassive neutron star,
    \todo{cite}
    more massive than the fundamental
    limit discussed above ($M\lesssim2\,M_\odot$), but nevertheless quasi-stable
    because it is rotating rapidly.
    If so, the supermassive neutron star may emit magnetized winds that shock
    into each other and emit thermal x-ray radiation. (Kumar-Rezolla 2015).
    \todo{fix cite, add Metzger+}
    \todo{cite observations}
  \item Gamma ray burst: If a nuclear accretion disk forms---either after the
    spin-down of the supermassive neutron star, or because a black hole forms
    immediately after merger, or because the progenitor is a neutron star--black
    hole \nsbh binary---some combination of magnetic and neutrino heating
    could drive an ultrarelativistic jet. The jet, when it shocks into the
    circumburst medium, or shocks internally, will emit high energy gamma ray
    photons, highly beamed in the direction of the jet, due to its relativistic
    speed. The burst of gamma ray emission lasts as long as the accretion is
    active enough to drive the jet: about a second.

    Many gamma ray bursts have
    been observed, and fall roughly into two categories, long and short,
    \todo{cite observations}
    according to their length. The observed short gamma ray bursts have
    properties that are well-modeled by this merger--disk--jet scenario,
    \todo{cite old proposal paper}
    and the astronomical community widely accepts the explanation.
    \todo{cite widely accepts}
    \todo{cite sgrbs}
\end{itemize}

Some of these electromagnetic signals are strong---in particular gamma ray
bursts---and our observing horizon encompasses many millions? of galaxies.
\todo{estimate this}
With space-based telescopes like Swift?, Fermi?, and INTEGRAL?
tens of short gamma ray bursts are observed per year.
\todo{check/cite Nakar 2007}
In an indirect sense, this rate gives us an observational answer to the question,
``how common are neutron star--black hole mergers?''.
But in the next few years a much more direct answer will be possible, as the
advanced generation gravitational wave interferometers, LIGO and Virgo, begin to
listen for the signature chirps of compact binary mergers. Unless we are greatly
mistaken about compact binaries, these observatories will likely pick up
? neutron star--neutron star \nsns,
? neutron star--black hole \nsbh, and
? black hole black hole \bhbh mergers per year.
\todo{cite Abadie+}

Of course it is possible the astrophysics community is mistaken.
The physical processes linking compact binary mergers to these
signals are not fully understood, an ignorance that partly motivates this thesis.
In fact, until astronomers observe one of these electromagnetic signals in
coincidence with a gravitational wave signal, it's possible (though very hard
for this physicist to imagine) that the short
gamma ray burst rate---providing the observational answer to our question
above---has nothing to do with mergers of compact binaries of any kind.

\subsubsection{Focusing in on Neutron Star--Black Hole Mergers}
Though much in the next sections is applicable to any type of merger involving
a neutron star, let us focus in on neutron star--black hole \nsbh
mergers, the topic of this thesis. Sec.~\ref{sec:why_this_model} defends this
focus more extensively than here.
But briefly: in this study, we are interested in systems involving the
strongest gravitational effects and nuclear matter. The first interest leads us
to black holes, and the second interest leads us to neutron stars.
We won't pursue neutron star--neutron star
\nsns or black hole--black hole \bhbh binaries past this section.

\subsection{What Roles May Neutrinos Play?}
\label{ssec:neutrino_roles}

%********************************************************************************
% Remnant stuff from pre-first-draft

However, neutrinos are freely produced in the neutron-rich matter. The
interactions that dominate neutrino opacity are scattering and absorption onto
nucleons and nuclei \citep[Sec.\ 11.7]{shap1983-bh_wd_ns}, all of which processes
scale like $\sigma_0(\varepsilon/m_e c^2)^2$. Here $\sigma_0$ is the weak
scattering cross section,
\begin{align}
  \sigma_0
  &\equiv \frac{4}{\pi}\left(\frac{\hbar}{m_e c}\right)^{-4}
  \left(\frac{G_F}{m_e c^2}\right)^2 \\
  &\sim   1.76 \times 10^{-44} \,\, {\rm cm}^2.
\end{align}
(See, for example, \citealt{tubb1975-neutrino_opacities}.)
%********************************************************************************

\section{What's the History Behind this Study?}
\label{sec:history}

\section{Why this Particular Model?}
\label{sec:why_this_model}

\section{What's the Scope and Trajectory of this Thesis?}
\label{sec:scope}


% Introduction
\chapter{Introduction}
\label{chap:intro}

A massive star that has exhausted its nuclear fuel, ends its radiative life
in a massive supernova, leaving behind either a hot, dense neutron star supported
against gravitational collapse by neutron degeneracy pressure, or a black hole.

Mergers of neutron stars and black holes can yield short-lived nuclear accretion
disks, which emit copious neutrinos. These neutrinos cool the disk and change
its composition over a timescale of milliseconds. We develop a model of the
major effects of this radiation on the disk in Chap.~\ref{chap:leakage}
\citep{deat2013-leakage} and characterize its neutrino emission in
Chap.~\ref{chap:ray_tracing}.

% Notation Conventions
% Brett Deaton -- Spring 2015

\section{Conventions}
\label{sec:conventions}

I adopt the standard general relativists metric signature convention
$(-,+,+,+)$.
Unless otherwise noted, I display equations in natural geometric units, so that
Newton's gravitational constant, $G$, and the speed of light, $c$, have unit
value.
\todo{not true, be consistent by chapter, and note it here}

I use abstract tensor notation in general: a tensor typeset with its indeces
represents the geometric object in all of its coordinate freedom. In other words
$T^{\alpha \beta}$ is identical to $T$, and is not $T$'s $\alpha\beta$-th
component.
In some equations, I use coordinate index notation accompanied with a warning.
Spatial tensors are given indices from the Latin alphabet ($a,b...$);
they exist on a 3-dimensional manifold. Spacetime tensors are given indices from
the Greek alphabet ($\alpha,\beta...$); they exist on a 4-dimensional manifold.
In instances of coordinate index notation, it follows naturally that
$\alpha,\beta...\in\{1,2,3\}$ and $a,b...\in\{0,1,2,3\}$.

Of course some symbols have labels other than their tensor indeces.
In particular, I use subscript labels before tensor indeces (e.g.\ the
momentum of a neutrino is $p_{\nu \alpha}$). I allow context to clarify when
an index ranges over some set other than dimension (e.g.\ the momentum of the
i-th flavor of neutrino is $p_{\nu_i \alpha}$). The subscript $\nu$ is always
used as a label, not as a tensor index.

\begin{table}
  \centering
  \begin{tabular}{rll}
    \textbf{Symbol}       & \textbf{Object}           & \\%\textbf{Additional Notes} \\
    $n_b$                 & baryon number density     & \\
    $\rho$                & rest density              & $\rho \equiv m_U n_b $ \\
    $\epsilon$            & specific internal energy  & not including the mass energy \\
    $h$                   & specific enthalpy         & $h=1+\epsilon+P/\rho$\\
    $P$                   & pressure                  & \\
    $T$                   & temperature               & \\
    $T^{\alpha \beta}$    & stress-energy tensor      & \\
    $Y_e$                 & electron fraction         & $Y_e=n_e/n_b$\\
    $\rho_*$              & density evolution variable& $\rho_*=\sqrt{g}W\rho$ \\
    $\tilde\tau$          & energy evolution variable & $\tilde\tau=\rho_*(hW-1)-\sqrt{g}P$ \\
    $\tilde S$            & momentum evolution variable & $\tilde S=\rho_*hu_i$\\
    $v^i$                 & Lagrangian velocity       & a.k.a.\ transport velocity, $v^i=u^i/u^t$ \\
    $u^\alpha$            & fluid four-velocity       & $u^a=dx^a/d\tau$ \\
    $U^\alpha$            & observer four-velocity    & used if we want to distinguish from $u^\alpha$ \\
    $d\tau$               & spacetime interval        & $d\tau=-dx^a dx^b \psi_{ab}$ \\
    $W$                   & Lorentz factor            & $\alpha u^t$ (if no gravity $W=1/\sqrt{1-v^2}$) \\
    $\tau$                & optical depth             & \\
    $Q_\nu$               & local energy emission rate& \\
    $R_\nu$               & local lepton number rate  & \\
    $\mu$                 & chemical potential        & $\mu=T\eta$ \\
    $p_\alpha$            & momentum 1-form           & \\
    $\varepsilon$         & neutrino energy           & used for asymptotic and local \\
    $\psi_{\alpha\beta}$  & spacetime metric          & \\
    $g_{ij}$              & spatial metric            & \\
    $\beta^i$             & shift                     & where $\beta_i = \beta^j g_{ij}$ \\
    $\alpha$              & lapse                     & \\
  \end{tabular}
  \caption[Symbols used in the text]{
    Some symbols used in this text.
    Indices are used in the ``Symbol'' column in an abstract sense, and in the
    ``Additional Notes'' column in a component sense.
  }
  \label{tab:conventions}
\end{table}

\begin{table}
  \centering
  \begin{tabular}{rlll}
    \textbf{Symbol} & \textbf{Value}       & \textbf{Units}           & \textbf{Description} \\
    $G$             & $6.67\times10^{-8}$  & cm$^3$ g$^{-1}$ s$^{-2}$ & Newton's gravitational constant \\
    $c$             & $3.00\times10^{10}$  & cm s$^{-1}$              & speed of light in vacuum \\
    $k_{\rm B}$     & $8.62\times10^{-11}$ & MeV K$^{-1}$             & Boltzmann's constant \\
    $G_{\rm F}$     & $2.3\times10^{-22}$  & cm MeV$^{-1}$            & Fermi coupling constant \\
    $G_{\rm F}/(\hbar c)^3$ & $1.17\times10^{-11}$ & MeV$^{-2}$       & ", naturalized units, $\hbar=c=1$ \\
    $\sin^2\theta_w$& 0.231                &                          & weak-mixing angle \\
    $m_e$           & 0.511                & MeV c$^{-2}$             & mass of electron \\
    $m_U$           & 939                  & MeV c$^{-2}$             & average nucleon mass, $(m_n+m_p)/2$ \\
    $M_\odot$       & $2.00\times10^{33}$  & g                        & solar mass \\
    $\hbar$         & $6.58\times10^{-22}$ & MeV s                    & reduced Planck's constant, $h/2\pi$ \\
  \end{tabular}
  \caption[Physical constants used in the text]{
    Some physical constants used in this text. Taken from the Particle Data
    Group, Particle Physics Booklet \citep{oliv2014-pdg}.
    I am only interested in scales, so I use three significant figures here.
    Note, the conversion factor between $G_{\rm F}$ in cgs and
    naturalized units is the naturalized length scale,
    $\hbar c/\varepsilon=1.97\times10^{-11}/(\varepsilon/{\rm MeV})$~cm.
  }
  \label{tab:constants}
\end{table}


\section{Timescales}

\section{General Relativistic Hydrodynamics}

\subsubsection{Foliating Spacetime}
\label{ssec:adm_metric}
In everything that follows we employ the 3+1 splitting of the spacetime metric,
\todo{cite Arnowitt et al.\ 1962}
$ds^2 \equiv \alpha^2dt^2 + g_{ij}(dx^i+\beta^idt)(dx^j+\beta^jdt)$.
The cartesian decomposition of the metric and its inverse is
\begin{equation}
  \psi_{\mu\gamma} :=
  \left(
  \begin{matrix}
    -\alpha^2 + \beta^i \beta_i  & \beta_i \\
    \beta_j                      & g_{ij}
  \end{matrix}
  \right)
  \qquad
  \psi^{\mu\gamma} :=
  \left(
  \begin{matrix}
    -\frac{1}{\alpha^2}          & \frac{\beta^j}{\alpha^2} \\
    \frac{\beta^i}{\alpha^2}     & g^{ij} - \frac{\beta^i \beta^j}{\alpha^2}
  \end{matrix}
  \right)
\end{equation}
where $\beta_i=\beta^j g_{ij}$, and $g^{ij}g_{ik}=\delta^{j}_{k}$.
This decomposition generates a foliation of the spacetime manifold into
three-dimensional spacelike submanifolds, whose coordinates are related
by the timelike congruence $n^\mu=dx^\mu/\alpha dt$, or in cartesian
decomposition, $n^\mu:=(1/\alpha,-\beta^i/\alpha)$.

\section{Radiation Transport}


% Leakage
\chapter{Effect of Neutrino Emission on the Accretion Disk}
\label{chap:leakage}

\section{Numerical Methods for Neutrino Leakage}

\section{Trustworthiness and Error}

\section{Dynamical Outflows}

\section{Accretion}

\section{Dominant Effect of Neutrinos on Accretion}


% Ray Tracing
\chapter{Properties of Neutrino Emission}
\label{chap:ray_tracing}

The leakage approximation used in Chap.~\ref{chap:leakage} gives us a beginning
understanding of the cooling effects of neutrinos on the accretion disk, or
how the disk changes as it radiates. Through that approximation we have been able
to estimate changes in the energy and composition of the fluid at the position
$x^\alpha$ due to the emission of neutrinos at $x^\alpha$.

However, the leakage approximation teaches us very little about the properties of
that emission. We \emph{can} extract some rough global measures:
1) the sum of the energy losses from all positions in the disk (i.e.\
the volume integral of $Q_{\nu_i}$ defined in Sec.~\ref{sec:leakage})
gives us the total energy luminosity, $L_{\nu_i}$, at any epoch,
2) likewise the sum of $R_{\nu_i}$ gives us the total lepton number luminosity
(also denoted $R_{\nu_i}$),
3) and the ratio of these two luminosities gives us the average neutrino energy,
$\langle \varepsilon_{\nu_i} \rangle$. These three calculations are presented in
Fig.~\ref{fig:neutrinos_by_species}.

But to answer some of the questions presented in Chap.~\ref{chap:overview}, we
need to know more about the radiation field outside of the disk, particularly
how it varies with position, $x^\alpha$, and how it is distributed in energy and
angle, $p_k$.
That is, we need to develop an estimate of the neutrino distribution function,
$f_{\nu_i}(x^\alpha;p_k)$.

Where the mean free path of the neutrinos is long with respect to local fluid
\todo{show this is the case}
scales at $x^\alpha$, $f(x^\alpha)$ is dependent upon the state of matter far
away from $x^\alpha$. In lieue of solving the Boltzmann equation in full
\todo{ref eqn in Chap.~\ref{chap:intro}}
(which would involve an evolution of a scalar field like those presented in
Chap.~\ref{chap:leakage} for fluid and metric fields, but over a 6-dimensional
phase space manifold with coordinates $\{x^j,p_k\}$, and therefore
computationally out of reach today),
we can capture this nonlocality with a ray tracing solution.
In ray tracing, $f(x^\alpha;p_k)$ is computed by tracing a neutrino trajectory
backwards from $x^\alpha$ with a momentum $p_k$, keeping track of additions
and depletions to the neutrino number density along that trajectory.
Each ray yields an approximate solution to the Boltzmann equation at a
single point on the phase space manifold of interest. A ray
tracing solution to the Boltzmann equation places an observer at $x^\alpha$,
points him in the direction $p_k$, and asks ``how many neutrinos from that
direction?''
This approximation neglects additions to the neutrino number density due to
scattering into the line of sight (the kind of scattering that gives to the
moon or a streetlamp a halo on a foggy night).
\todo{ref later discussion}

In this chapter I present a ray tracing algorithm for estimating
$f_{\nu_i}(x^\alpha;p_k)$ in relativistic numerical spacetimes and fluid
distributions, I calculate $f_{\nu_i}$ describing neutrino
emission from the model accretion disk from Chap.~\ref{chap:leakage},
and I use these $f$s to estimate $q_{\nu\bar{\nu}}$, the heating due to
neutrino-antineutrino annihilation in the funnel of the disk.
Because of large uncertainties,
this last calculation should be considered a proof-of-concept for future
calculations. It may also be considered a back-of-the-envelope answer to
the question, ``Can a neutron star--black hole merger drive a gamma ray burst
jet by neutrino processes alone?''

Sec.~\ref{sec:f_algorithm} gives a technical overview of the ray tracing
algorithm and two code tests. Sec.~\ref{sec:f_this_case} presents the distribution
function of the three 'leakage' neutrino species around this model accretion
disk. Secs.~\ref{sec:q_algorithm}~and~\ref{sec:q_this_case} describe the
algorithm for and results from a neutrino-antineutrino annihilation heating
calculation for this model.

\section{Calculating $f_\nu$: Ray Tracing Algorithm}
\label{sec:f_algorithm}

In the most comprehensive treatment, a ray tracing solution for $f$ is calculated
by solving the rendering equation backwards along a trajectory terminating at
$x^a$ with momentum $p_k$:
\todo{derive rendering eqn, or check/ref below}
\begin{equation}
  \label{eqn:rendering}
  f(x^\alpha;p_k) = \int_{x^a}^{\rm far\,away} \diff \lambda \, \mathcal{E}(p_k)
  \exp\left(-\int_{x^a}^\lambda  \diff \lambda' \, \mathcal{A}(p_k)\right)
\end{equation}
where $\mathcal{E}$ and $\mathcal{A}$ are the invariant emissivity and opacity
defined in Chap.~\ref{chap:intro}.
\todo{ref eqns}
For isotropic scattering, $p_k$ only enters $\mathcal{E}$ and $\mathcal{A}$ in the
form $-p_\beta u^\beta$, the neutrino energy in the rest frame of the fluid.
Furthermore, $p_\beta$ may be calculated from $p_i$ by invoking the neutrino mass,
$-m_{\nu_i}^2=p_\beta p^\beta$.
All the neutrino masses are vanishingly small in the case of nuclear
accretion disks, with fluid energy scales of a few MeV. So in the following we
set $m_{\nu_i}=0$.

As $f$ accumulates along the ray, the integrand in Eqn.~\ref{eqn:rendering}
gets attenuated by the exponential term, which is the optical depth defined in
Chap.~\ref{chap:intro}.
\todo{ref eqn}
At large optical depths, the integrand vanishes exponentially, because fluid
properties on the far side of opaque matter cannot affect the neutrino
distribution function on this side. The exponential behavior of the integrand
leads us to pose a further approximation, that all neutrinos are emitted from
a single point along the trajectory, the neutrinosurface.

In this treatment, the ray is traced backwards until the optical depth becomes
large ($\tau\sim1$). This defines the neutrinosurface at $x_0^\alpha$. The
neutrinos are assumed to decouple from the matter at this surface, and so f
is simply the Fermi-Dirac distribution:
\begin{equation}
  \label{eqn:f_fermi_dirac}
  f(x^\alpha;p_k) =
  \left(\exp\left(\frac{-p_\beta u^\beta(x_0^\alpha)}{T(x_0^\alpha)}
  -\eta(x_0^\alpha) \right)+1\right)^{-1}
\end{equation}
where $T(x_0^\alpha)$ is the fluid temperature at the neutrinosurface, and
$\eta(x_0^\alpha)$ is the neutrino chemical potential scaled by $T$.
We use $\eta$ estimated by the leakage scheme (Eqn.~\ref{mu_nu}).
\todo{other choices for $\eta$}

Because neutrino scattering processes scale with
$\varepsilon^2=(-p_{\nu}^\beta u^\beta)^2$,
\todo{ref intro}
we use the energy-factored opacity $\zeta=\chi/\varepsilon^2$, where $\chi$ is
the cumulative opacity:
\begin{equation}
  \chi = \sum\limits_{{\rm processes}\, i} \chi_i.
\end{equation}
We consider scattering by nucleons and nuclei and absorption onto nucleons,
as described in Sec.~\ref{sec:leakage}.
Because the fluid and spacetime configuration at this late time is approximately
axisymmetric and stationary on the timescale of a neutrino-crossing time
($\sym1$~ms), we trace our geodesics through a single time slice of data,
rather than taking timesteps backwards through previous data slices,
interpolating in between slices. This greatly reduces the memory demands (each
time slice comprises $\sym$1~GB of data) and complexity of the calculation.
\todo{estimate disk changes over 1~ms: from response to leakage referee}

\subsection{Geodesic Equations}
\label{ssec:geodesic}
In a general spacetime like that of our disk model, neutrinos do not follow
straight Euclidean paths, but curves that obey the geodesic equation,
\begin{equation}
  \label{eqn:geodesic}
  0=\frac{d^2x^\alpha}{d\lambda^2} + \Gamma^\beta_{\alpha\gamma}
  \frac{dx^\alpha}{d\lambda}\frac{dx^\gamma}{d\lambda},
\end{equation}
This second-order equation may be split into two coupled first-order equations
by choosing the affine parameterization $\lambda$ such that
$dx^\alpha/d\lambda=p^\alpha$ and
$dp^\beta/d\lambda=\Gamma^\beta_{\alpha\gamma}p^\alpha p^\gamma$.
\todo{umm, check p eqn}
The connection coefficients may be calculated from derivatives of the metric,
\begin{eqnarray}
  \label{eqn:christoffel}
  \Gamma^\beta_{\alpha\gamma}
  &=& \psi^{\beta\mu}\Gamma_{\mu\alpha\gamma} \nonumber \\
  &=& \frac{1}{2} \psi^{\beta\mu}
  (\psi_{\mu\alpha,\gamma} + \psi_{\mu\gamma,\alpha} - \psi_{\alpha\gamma,\mu}).
\end{eqnarray}

Neutrinos have rest masses less than a few eV \citep{oliv2014-pdg}.
Therefore, at the neutrino energy scale
of this problem (a few MeV, see Fig.~\ref{fig:neutrinos_by_species}), the
geodesics are essentially null.
\todo{say why this matters}

We integrate the geodesic equations parameterized by coordinate time,
using the formulation introduced by \cite{hugh1994-eh_finding}.
The extensive framework to integrate the \citeauthor{hugh1994-eh_finding}\
equations through a numerical spacetime demands efficient handling of
pointwise interpolation of metric fields to arbitrary positions, reading from
disk and reconstructing into memory the metric data dumped from a previous
simulation, and accurate integration of a system of ordinary differential
equations.
All of this was implemented by a graduate student at Cornell, Andy Bohn, in order
to find event horizons and compute gravitational lensing; the latter was
published in \cite{bohn2015-lensing}.
I piggy backed my ray tracing code on top of his general and efficient framework.

In addition to equations for $x^j$ and $p_k$, we integrate $\tau$, the optical
depth along the ray. Thus our entire system of equations is
\begin{eqnarray}
  \label{eqn:geo_x}
  \frac{dx^j}{dt} &=& g^{ji}\frac{p_i}{p^t} - \beta^j \\
  \label{eqn:geo_p}
  \frac{dp_k}{dt} &=& -\alpha \alpha_{,k}p^t + \beta^i_{,k}p_i
  - \frac{1}{2}g^{ji}_{,k} \frac{p_jp_i}{p^t} \\
  \label{eqn:geo_tau}
  \frac{d\tau}{dt} &=& \frac{\varepsilon^3}{p^t} \zeta,
\end{eqnarray}
where $\alpha$ is the lapse, $\beta^j$ the shift, $g^{ij}$ the inverse
induced metric on the spatial slice, and $\varepsilon$ is the neutrino
energy in the frame comoving with the fluid.

A word on $p^t$, used in Eqns.~\ref{eqn:geo_x}--\ref{eqn:geo_tau}.
As is well-known, $p_t$ does not change along a geodesic trajectory if the
spacetime is stationary. (This result may be derived directly from the geodesic
equation, Eqn.~\ref{eqn:geodesic}.)
However, $p^t$, in general, \emph{does} change. We may calculate it in the
following steps: 1) enforce $p_\alpha p^\alpha=0$ by calculating
$p_t=p_i\beta^i-\alpha\sqrt{p_ip_jg^{ij}}$, so that all four components of the
momentum 1-form are known, then 2) calculate the time-component of the dual of
this 1-form, $p^t=\psi^{t\alpha}p_\alpha$, where $\psi^{\alpha\beta}$ may be
computed from the lapse, shift, and induced spatial metric by matrix inversion.

A word on $\varepsilon=-p_\alpha u^\alpha$, used in Eqn.~\ref{eqn:geo_tau}.
In our simulations we don't store $u^\alpha$ directly, but rather the spatial
components of the fluid four-velocity 1-form, $u_i$, and the fluid Lorentz
factor, $W$. (The Lorentz factor doesn't need to be stored, since it may
be calculated from the normalization of $u^\alpha$, by
$W=(u_iu_jg^{ij}+1)^{1/2}$. However it is used throughout our code in enough
places that storing $W$ improves our code's efficiency.)
From these variables, we may calculate
\begin{equation}
  \varepsilon = -p_t W/\alpha -p_i(u_j g^{ij}-\beta^i W/\alpha).
\end{equation}

\subsection{Momentum Conventions}
\label{ssec:p_conventions}
Because of the natural symmetries of our spacetime, it is more convenient to use
momentum components in spherical polar than in cartesian representation. We
describe the map between these representations here.

The neutrino's 4-momentum is fully specified by three numbers, the fourth being
constrained by the mass of the neutrino, which vanishes with respect to the
energies in this problem. Instead of using three of the spherical polar
components, we define a Euclidean magnitude, $p$, and two angles $\alpha$,
$\beta$ which map to the cartesian components of the covariant 4-momentum by
the standard spherical polar map
\begin{align}
  &p_x = p\sin\alpha\cos\beta \\
  &p_y = p\sin\alpha\sin\beta \\
  &p_z = p\cos\alpha,
\end{align}
so that $p = (p_x^2+p_y^2+p_z^2)^{1/2}$.
Fig.~\ref{fig:p_conventions} provides a visual reference for these definitions.

\begin{figure}
  \centering
  \begin{subfigure}{.5\textwidth}
    \centering
    \includegraphics[width=1\linewidth]{Figures/spherical_polar_map}
  \end{subfigure}
  \begin{subfigure}{.45\textwidth}
    \centering
    \includegraphics[width=0.8\linewidth]{Figures/neutrino_momentum_conventions}
  \end{subfigure}
  \caption[Conventions for momentum components]{
    In both diagrams, the broad yellow arrow represents the neutrino's momentum.
    \emph{Left Figure}:
    Spatial basis vectors ($\vec{e}_i\equiv\partial/\partial x^i$)
    at a representative position on the manifold.
    The coordinates $\{r,\theta,\phi\}$ are defined with respect to
    $\{x,y,z\}$ by the standard spherical polar to cartesian map
    (e.g.\ $x=r\sin\theta\cos\phi$).
    The spherical basis vectors are blue, and the cartesian basis vectors black.
    \emph{Right Figure}:
    The cotangent space at the neutrino's position.
    The neutrino momentum 1-form, $p_\mu$,
    may be written componentwise using the basis dual to $\vec{e}_i$
    ($\underline{e}^i\equiv dx^i$).
    The neutrino momentum is completely specified by three numbers.
    We use $\{p_t$,$\alpha$,$\beta\}$, which describe the cartesian spatial
    components of $p_\mu$ with the standard spherical polar to cartesian map
    (e.g.\ $p_x=p\sin\alpha\cos\beta$).
    The $p$ in these formulae is the Euclidean magnitude,
    $p = (p_x^2+p_y^2+p_z^2)^{1/2}$, related to $p_t$ by
    Eqns.~\ref{eqn:p_to_pt}~and~\ref{eqn:angular_factor}.
    (Note, we draw the neutrino's momentum vector embedded in the
    spatial manifold in the left figure, even though it properly lives only in
    the cotangent space in the right figure.)
    }
  \label{fig:p_conventions}
\end{figure}

To clean up some future calculations, we also define a 1-form on the
spatial manifold, $\Omega_i$, so that
\begin{align}
  &\Omega_i :\equiv (\sin \alpha \cos \beta, \sin \alpha \sin \beta, \cos \alpha) \\
  &p_i = p\,\Omega_i.
\end{align}

The fourth component of the momentum is constrained by the null condition,
$0=\psi^{\mu\gamma}p_\mu p_\gamma$.
In the 3+1 foliation of spacetime, we have
\begin{equation}
  \psi^{\mu\gamma}p_\mu p_\gamma
  = -\frac{1}{\alpha^2} p_t^2 + \frac{2}{\alpha^2} \beta^i \Omega_i p\, p_t
  + (g^{ij}-\frac{\beta^i \beta^j}{\alpha^2}) \Omega_i \Omega_j p^2, \nonumber
\end{equation}
whose solution is
\begin{equation}
  \label{eqn:p_to_pt}
  p_t = C(\Omega_i) \, p
\end{equation}
\begin{equation}
  \label{eqn:angular_factor}
  C(\Omega_i) \equiv \beta^i \Omega_i \pm
  \alpha \sqrt{g^{ij} \Omega_i \Omega_j}.
\end{equation}

As a check it is easy to confirm that for
$\psi_{\mu \nu}=\eta_{\mu \nu}:=\text{diag}(-1,1,1,1)$,
$C \rightarrow -1$, recovering the flat space connection between time and space
components of momentum. This check also confirms that we use the minus sign in
Eqn.~\ref{eqn:angular_factor}.

In some cases I will break from this convention with a simple rotation, in order
to have a momentum basis oriented with respect to the radial direcion, $\hat{r}$.
I label momentum colatitude and azimuthal angles in the rotated frame as $A$
and $B$ respectively. These angles are related to $\alpha$ and $\beta$ by a
simple Euler rotation, the specifics of which are irrelevant to this thesis.

\subsection{Numerical Integration}
\label{ssec:timestepping}
Eqns.~\ref{eqn:geo_x}--\ref{eqn:geo_tau} are solved by discretizing time and
performing a numerical integration.
I describe that process here.

A first-order ordinary differential equation has the form
\begin{equation}
  \partial_t u = L(u),
\end{equation}
which may be discretized into timesteps.
The value of $u$ at the $n$-th timestep is updated to its value
$\Delta t$ later by a fifth-order-accurate Dormand-Prince algorithm,
represented here abstractly as DP5[...]:
\todo{cite?}
\begin{equation}
  u^{n+1} = {\rm DP5}[L,u^n,\Delta t].
\end{equation}

As a high-order integration method, DP5 consists of a series of sub-steps.
Each sub-step evaluates the equation's right hand side, $L(u)$, using $u$ from
the previous step.
\todo{also dense output}
DP5 is an embedded Runge-Kutta algorithm, designed to also yield
a fourth-order-accurate solution with few additional evaluations of $L(u)$.
The difference in $u^{n+1}$ between the fourth- and fifth-order-accurate
solutions gives an estimate of the error for that step size. If this error is
larger than some threshold, we reduce the timestep, and try again from $u^n$.
\todo{Abs-Rel explicit}

The DP5 algorithm works as well for a system of coupled equations,
like the seven components of Eqns.~\ref{eqn:geo_x}--\ref{eqn:geo_tau}:
$\{x^x,x^y,x^z,p_x,p_y,p_z,\tau\}$.
In this case, the above prescription holds, but we think of $u$ as a
vector and $L$ as a vector-valued vector function.

We terminate the ray at the neutrinosurface, where $\tau=1$.
However, because a priori we do not know how much optical depth will accumulate
over a given timestep, we simply overstep the neutrinosurface, terminating at
the step N for which $\tau^{N-1}<1$ and $\tau^N\geq1$. Using the DP5 sub-steps
to interpolate $\tau$ to any time between $t^{N-1}$ and $t^{N}$, we search
this domain for the time $t_0$ that solves the equation
$\tau(t)=1$, using a robust Newton-Raphson root finding algorithm.
\todo{expected convergence order}
We examine the error associated with the uncertainty in the location of the
neutrinosurface in Sec.~\ref{sssc:af_sphere}.

\subsection{Code Tests}
\label{ssec:f_tests}
To test our code, we compute $f$ in two scenarios for which the analytic solution
is known.

\subsubsection{Homogeneous Minkowski}
In a homogeneous medium of temperature $T$, opacity $\zeta$, and neutrino chemical
potential $\eta$, on a flat spacetime manifold, $f$ is isotropic and homogeneous:
\begin{equation}
  f(x^\alpha;p_k) \rightarrow f(\varepsilon)
  =(e^{\varepsilon/k_{\rm B}T-\eta}+1)^{-1}
\end{equation}
The solution is scale-invariant because the observer sits inside of a
neutrinosurface which emits the same number of neutrinos per angle per energy
whether the surface is near (for high-energy rays) or far (for low-energy rays).

Despite this scale-invariance, we may calculate the length of a ray of energy
$\varepsilon=-p_t=p^t$ from Eqn.~\ref{eqn:geo_tau}:
$\ell \equiv c\Delta t = \Delta \tau/\zeta\varepsilon^2$, in order to test the
opacity integration by ray tracing code.

In our code test, we used volume data:
$T=5$~MeV,
$\zeta_{\nu_i}=\{0.01,0.005,0.001\}$~km$^{-1}$~MeV$^{-2}$, and
$\eta_{\nu_i}=\{0.1,-0.1,0\}$.
We sampled the distribution functions by tracing rays of energy
$\varepsilon=10$~MeV, and found neutrinosurfaces at distances of
$\ell_{\nu_i}=\{2.00,1.00,10.0\}$~km and neutrino number densities of
$f_{\nu_i}=\{0.109,0.130,0.119\}$ for each species respectively,
as expected. These results were exact to double precision because the fields
varied smoothly, and interpolation was exact.

\subsubsection{Asano-Fukuyama Sphere}
\label{sssc:af_sphere}
Outside of a hot, compact, non-rotating star in dynamical equilibrium, the
spacetime is spherically symmetric and stationary. The only solution of that
kind satisfying Einstein's equations is the Schwarzschild spacetime.
In Schwarzschild spherical polar coordinates, the metric takes the form
$\psi_{\mu\gamma}(r):={\rm diag}\left(-\alpha^2(r),\alpha^{-2}(r),r^2,r^2\sin^2\theta\right)$,
where the lapse is $\alpha(r)=(1-R_g/r)^{1/2}$, and the gravitating mass is
$M=R_g/2$.
(Note, this scenario was examined semi-analytically by Asano and Fukuyama (2000),
\todo{cite properly}
to estimate the neutrino-antineutrino annihilation power outside of a collapsed
stellar core.)

A neutrino is emitted from the neutrinosphere at $R_\nu$, with energy
$\varepsilon'$ in the fluid rest frame.
%(with $u^\beta:=(-\alpha(R_\nu),0,0,0)$, so that
%$p_{\nu t}=-\alpha(R_\nu)\varepsilon'$),
It will be measured by a stationary observer at $r$
%(for whom $U^\beta:=(-\alpha(r),0,0,0)$),
as having a lower energy
$\varepsilon=\varepsilon' \, \alpha(R_\nu)/\alpha(r)$.
Therefore, for momentum angles in the direction of the neutrinosphere,
$f(\varepsilon)=\left(\exp(\varepsilon/k_{\rm B}T_{\rm eff}-\eta)+1\right)^{1/2}$,
with an effective temperature $T_{\rm eff}=T \, \alpha(R_\nu)/\alpha(r)$,
where $T$ is the fluid temperature measured in the fluid rest frame.
In terms of the conventional momentum variables $\{p_t,\alpha,\beta\}$
introduced in Sec.~\ref{ssec:p_conventions}, we have
\begin{equation}
  \label{eqn:asano_fukuyama_f}
  f(x^\alpha;p_k) \rightarrow f(r;p_t,A)
  =\left\{
  \begin{matrix}
    (e^{-p_t/k_{\rm B}T\sqrt{(1-R_g/R_\nu)}-\eta}+1)^{-1} & A \leq A_{\rm max}(r) \\
    0                                                            & A > A_{\rm max}(r)
  \end{matrix}
  \right.
\end{equation}
where $A$ is the angle $p_k$ makes with a radial ray, and
$A_{\rm max}$ is the half-angular size of the star as viewed from $r$.
In a flat spacetime it's simple: $\sin A_{\rm max} = R_\nu/r$.
But in the Schwarzschild spacetime, where neutrino trajectories bend around the
star,
\begin{equation}
  \label{eqn:angular_extent}
  \sin A_{\rm max} = \frac{R_\nu}{r} \sqrt{\frac{1-R_g/r}{1-R_g/R_\nu}},
\end{equation}
as derived by Asano-Fukuyama (2000).
\todo{cite properly, and fix, this isn't your angle $A$}

Fig.~\ref{fig:f_hot_ns} shows $f(p_t=20\,{\rm MeV})$ across all angles for
a patch of sky centered on the emitter. The spacetime and the emitter are
characterized by $T=5$~MeV, $\eta=0.1$, $R_g=2\,M_\odot$, $R_\nu=3\,M_\odot$.
To impose a sharp neutrinosphere at $R_\nu$, we use a step function opacity
with $\zeta=10^6\,{\rm Msun}^{-1}\,{\rm MeV}^{-2}$ inside the star.
This scenario represents an unphysically compact star (in fact, this
neutrinosphere coincides with the radius of circular neutrino orbits), but it
serves its purpose here to stress-test the ray tracing code in strong gravity.
For comparison, in the same figure we also show a ray tracing sampling of $f$
at the same coordinate viewing position for the same star embedded in flatspace.
Both the angular extent of the stars in Fig.~\ref{fig:f_hot_ns}, and the
intensity of neutrinos emitted by their surfaces at this energy agree
with the analytical predictions given by
Eqns.~\ref{eqn:angular_extent}~and~\ref{eqn:asano_fukuyama_f}.

For this test we use a fixed timestep, $\Delta t = 0.2\,M_\odot$,
because the adaptive timestepper described in Sec.~\ref{ssec:timestepping} is
not very effective with discontinuous fields, like $\zeta$. In physical
fluid configurations, we expect to encounter discontinuities also (though
probably not as extreme), and we may have to used fixed timesteps in those
cases also.

In Fig.~\ref{fig:f_hot_ns}, the Schwarzschild case, the variation in $f$
across the surface of the star is a spurious artifact of our finite timestep:
the redshift is greater along directions whose rays yield an estimate of the
neutrinosurface at $r<R_\nu$, and lesser for neutrinosurfaces at $r>R_\nu$.

\begin{figure}
  \centering
  \begin{subfigure}{.8\textwidth}
    \centering
    \includegraphics[width=1\linewidth]{Figures/fnue_Alpha_vs_Beta-asano_fukuyama_gr}
  \end{subfigure}
  \begin{subfigure}{.8\textwidth}
    \centering
    \includegraphics[width=1\linewidth]{Figures/fnue_Alpha_vs_Beta-asano_fukuyama_flat}
  \end{subfigure}
  \caption[Monochromatic sky map of $f$ for the hot compact star]{
    $f$ across a patch of sky in a single energy band, $p_t=20$~MeV.
    The source is a hot compact star: $R_g=2\,M_\odot$, $R_\nu=3\,M_\odot$,
    $T=5$~MeV, and $\eta=0.1$.
    The observer is stationary at Schwarzschild radius $5\,M_\odot$.
    \emph{Top Figure}: Schwarzschild spacetime.
    \emph{Bottom Figure}: Minkowski spacetime.
    These two stars have identical coordinate radii, and identical local fluid
    properties at their neutrinospheres.
  }
  \label{fig:f_hot_ns}
\end{figure}

Fig.~\ref{fig:f_spectrum_hot_ns} presents the energy spectrum emitted by this
surface over the broad band $p_t\in[0.01,100]~MeV$, for both Schwarzschild
and Minkowski cases.
We also show the relative errors in $f(p_t)$, with respect to the analytic
solution $f_{\rm ex}(p_t)$. The Minkowski calculation is exact to machine
precision, because location of the neutrinosurface has no effect on $f$.
But the relative errors in the Schwarzschild case are large, and grow to
20\% at 100~MeV. (It is important to keep in mind that this is not the case
generally, but only when the neutrinosurface is in a steep gravitational
potential.
This will be an important consideration in Sec.~\ref{sec:q_algorithm}:
spectral information at energies near 100~MeV can be important in the
neutrino-antineutrino annihlation calculation because the process depends very
steeply upon the neutrino energies.

\begin{figure}
  \centering
  \begin{subfigure}{.7\textwidth}
    \centering
    \includegraphics[width=1\linewidth]{Figures/fnue_vs_E-asano_fukuyama}
  \end{subfigure}
  \begin{subfigure}{.7\textwidth}
    \centering
    \includegraphics[width=1\linewidth]{Figures/fnueError_vs_E-asano_fukuyama}
  \end{subfigure}
  \caption[Asymptotic energy spectrum of the hot compact star]{
    The $p_t$ spectrum of the hot compact star from the same viewing
    position as in Fig.~\ref{fig:f_hot_ns}. These plots show $f$ sampled by
    the $\cos A=0$, $B=0$ rays.
    \emph{Top Figure}: $f(p_t)$ in the case of a curved and flat spacetime.
    The exact distribution functions $f_{\rm ex}$ are represented as solid lines,
    which lie directly beneath the data points.
    \emph{Bottom Figure}: The relative error in $f$ for these two cases.
    For the flat spacetime distribution function, the error bottoms out at
    machine roundoff.
  }
  \label{fig:f_spectrum_hot_ns}
\end{figure}

\section{$f_\nu$ for this Accretion Disk}
\label{sec:f_this_case}

Let us examine the neutrino emission from the post-merger accretion disk model
presented in Chap.~\ref{chap:leakage}. Here we sample $f$ at two positions,
on the rotation axis and on the equator, about 120~km from the black hole
and at a late time, about 40~ms after merger. At this time the disk has reached
a quasi-equilibrium accretion state, with most of the asymmetry of the merger
having been smeared out by differential rotation, making the model ideal for our
time-independent ray tracing approximation. The disk's luminosity in neutrinos
has decayed to approximately $10^{53}$~erg~s$^{-1}$
(Fig.~\ref{fig:globalevolution}). In this sense, our later calculation of
neutrino-antineutrino annihilation power will provide a lower bound for energy
available to drive a jet.

In fact, 40~ms after a realistic merger with these bulk parameters, we expect
magnetic effects to play a significant role in hte accretion dynamics,
\todo{cite someone}
increasing the accretion rate via the magneto-rotational instability, and heating
the disk through a turbulent transport of energy from large scales (bulk kinetic
energy) to small scales (random thermal energy). In thismodel we neglect this
important physics, and again, in this sense, our later neutrino-antineutrino
annihilation power calculation will provide a lower bound on jet energetics.

%\begin{figure}
%  \centering
%  \begin{subfigure}{.7\textwidth}
%    \centering
%    \includegraphics[width=1\linewidth]{Figures/}
%  \end{subfigure}
%  \begin{subfigure}{.7\textwidth}
%    \centering
%    \includegraphics[width=1\linewidth]{Figures/}
%  \end{subfigure}
%  \caption[Accretion disk emission as seen from rotation axis]{
%  }
%  \label{fig:f_disk_axis}
%\end{figure}

\section{Calculating $q_{\nu \bar{\nu}}$: Integration Algorithm}
\label{sec:q_algorithm}

\section{$q_{\nu \bar{\nu}}$ for this Accretion Disk}
\label{sec:q_this_case}


\newpage

\bibliography{references}

\end{document}
